\documentclass{article}
\usepackage{h}

\title{Topology by James Munkres -- Chapter 1 Supplementary Exercises}
\author{William Gao}
\date{Summer 2024}

\begin{document}
\maketitle

A {\it topological group} $G$ is a group that is also a $T_1$ topological space such that $f: G \times G \rightarrow G$ defined by $x \times y \mapsto x \cdot y$ and $g: G \rightarrow G$ defined by $x \mapsto x^{-1}$ are continuous.

\begin{enumerate}
    \item Let $H$ be a group and a $T_1$ topological space. Show that $H$ is a topological group if and only if $h: H \times H \rightarrow H$ defined by $x \times y \mapsto x \cdot y^{-1}$ is continuous.
    
    {\bf SOLUTION.} Suppose $H$ is a topological group. Since $f, g$ are continuous, $h: H \times H \rightarrow H$ defined by $x \times y \rightarrow x \cdot y^{-1} = f(x \times g(y))$ is continuous by the composition and product of continuous functions. Conversely, if $h$ is continuous, then the restriction $h': \{e\} \times H \rightarrow H$ defined by $e \times y \rightarrow e \cdot y^{-1} = y^{-1}$ is continuous; this is precisely $g$. Moreover $f(x \times y) = h(x \times g(y))$ is continuous. $\Box$

    \item Show that the following are topological groups:
    \begin{enumerate}
        \item $(\ints, +)$.

        {\bf SOLUTION.} Since $\{n\}$ is open in $\ints$ for all $n \in \ints^+$, $\ints$ has the discrete topology and is clearly $T_1$. Moreover, every function in $\ints$ or $\ints \times \ints$ is continuous. $\Box$
        
        \item $(\reals, +)$.

        {\bf SOLUTION.} $\reals$ is Hausdorff and a group. $x^{-1} = -x$, so for any $x \times y \in \reals \times \reals$ and any neighbourhood $V$ of $h(x \times y) = x - y$, let $B(x - y, \varepsilon) \subseteq V$. Then $B_\rho(x \times y, \frac{\varepsilon}{2})$ is a neighbourhood of $x \times y$ such that if $a \times b \in B_\rho(x \times y, \frac{\varepsilon}{2})$ then
        $$\max\{|x - a|, |y - b|\} < \frac{\varepsilon}{2},$$
        and thus
        $$|(x - y) - (a - b)| \leq |x - a| + |y - b| < \varepsilon,$$
        meaning $h(a \times b) = a - b \in B(x - y, \varepsilon) \subseteq V$. Thus $h$ is continuous. $\Box$
        
        \item $(\reals^+, \cdot)$.

        {\bf SOLUTION.} $x^{-1} = \frac1x$. $h$ is simply the division operation, which we know is continuous on $\reals^+$. $\Box$
        
        \item $(S^1, \cdot)$ where $S^1 = \{z \in \comp: |z| = 1\}$.

        {\bf SOLUTION.} Since $\comp$ is Hausdorff and a group, so is $S^1$. Consider $h: S^1 \times S^1 \rightarrow S^1$ given by $x \times y \mapsto x y^{-1}$. Let $e^{i \alpha} \times e^{i \beta} \in S^1 \times S^1$ and let $V$ be a neighbourhood of $h(e^{i\alpha} \times e^{i\beta}) = e^{i(\alpha - \beta)}$ in $S^1$. There exists $\varepsilon > 0$ such that $e^{i(\alpha - \beta)} \in \{e^{i \delta}: \delta \in (\alpha - \beta - \varepsilon, \alpha - \beta + \varepsilon)\} \subseteq V$. Then $U = \{e^{i \delta}: \delta \in (\alpha - \frac{\varepsilon}{2}, \alpha + \frac{\varepsilon}{2})\} \times \{e^{i \delta}: \delta \in (\beta - \frac{\varepsilon}{2}, \beta + \frac{\varepsilon}{2})\}$ is a neighbourhood of $x \times y$ in $S^1 \times S^1$ and if $e^{ix} \times e^{iy} \in U$, then
        $$|(x-y )- (\alpha - \beta)| \leq |x - \alpha| + |y - \beta| < \varepsilon,$$
        hence $h(e^{ix} \times e^{iy}) = e^{i(x - y)} \in \{e^{i \delta}: \delta \in (\alpha - \beta - \varepsilon, \alpha - \beta + \varepsilon)\} \subseteq V$. Thus $h$ is continuous. $\Box$
        
        \item GL$(n)$ as a subset of $\reals^{n^2}$ under matrix multiplication.

        {\bf SOLUTION.} Since GL$(n) \subseteq \reals^{n^2}$, it is Hausdorff. $f$ is continuous since each component of $AB$ is a polynomial in the entries of $A$ and $B$; $g$ is continuous by Cramer's rule. $\Box$
    \end{enumerate}

    \item Let $H$ be a subspace of $G$. Show that if $H$ is a subgroup, then $H$ and $\overline{H}$ are topological groups.

    {\bf SOLUTION.} Since $H$ is a subspace, it is $T_1$. Moreover as $H$ is a subgroup, it is closed under $\cdot$ and inverses, so $\restr{f}{H}: H \times H \rightarrow H$ and $\restr{g}{H}: H \rightarrow H$ are continuous.

    $\overline{H}$ is similarly $T_1$. Since $f, g$ are continuous, we have $f(\overline{H} \times \overline{H}) = f(\overline{H \times H}) \subseteq \overline{f(H \times H)} = \overline{H}$ and $g(\overline{H}) \subseteq \overline{g(H)} = \overline{H}$. Thus $\restr{f}{\overline{H}}: \overline{H} \times \overline{H} \rightarrow \overline{H}$ and $\restr{g}{\overline{H}}: \overline{H} \rightarrow \overline{H}$ are continuous. $\Box$

    \item Let $\alpha \in G$. Show that $f_\alpha, g_\alpha: G \rightarrow G$ defined by
    $$f_\alpha(x) = \alpha \cdot x \text{ and } g_\alpha(x) = x \cdot \alpha$$
    are homeomorphisms of $G$. Conclude that $G$ is a homogeneous space.

    {\bf SOLUTION.} $f_\alpha, g_\alpha$ are the product of a constant function and an identity function composed with $x \times y \mapsto x \cdot y$. Thus they are continuous. Similarly their inverses $(f_\alpha)^{-1} = f_{\alpha^{-1}}, (g_\alpha)^{-1} = g_{\alpha^{-1}}$ are continuous, so they are homeomorphisms. Given $x, y \in G$, we will construct a homeomorphism $G \rightarrow G$ that maps $x$ to $y$ by $f_{y \cdot x^{-1}}$. Indeed, $$f_{y \cdot x^{-1}}(x) = (y \cdot x^{-1}) \cdot x = y \cdot (x^{-1} \cdot x) = y. \Box$$

    \item Let $H$ be a subgroup of $G$. If $x \in G$, define the left coset of $H$ in $G$ by $xH = \{x \cdot h: h \in H\}$. Let $G / H$ denote the collection of left cosets of $H$ in $G$; it is a partition of $G$. Give $G / H$ the quotient topology.
    \begin{enumerate}
        \item Show that if $\alpha \in G$, $f_\alpha$ induces a homeomorphism of $G / H$ carrying $xH$ to $(\alpha \cdot x) H$. Conclude that $G / H$ is a homogeneous space.

        {\bf SOLUTION.} We know $g = p \circ f_\alpha: G \rightarrow G / H$ defined by $x \mapsto (\alpha \cdot x)H$ is a quotient map since $p, f_\alpha$ are quotient maps. Moreover,
        $$g^{-1}(\{xH\}) = \{y \in G: (\alpha \cdot y)H = xH\}$$
        and $(\alpha \cdot y)H = xH$ implies for all $h_1 \in H$, $\alpha \cdot y \cdot h_1 = x \cdot h_2$ for some $h_2 \in H$. Then $y = \alpha^{-1} \cdot x \cdot h_2 \cdot h_1^{-1} \in (\alpha^{-1} \cdot x)H$. Conversely, if $y \in (\alpha^{-1} \cdot x)H$, then $\alpha \cdot y = x \cdot h$ for some $h \in H$. Then for any $h_1 \in H$, $\alpha \cdot y \cdot h_1 = x \cdot h \cdot h_1 \in xH$, and for any $h_2 \in H$, $x \cdot h_2 = x \cdot h \cdot h^{-1} \cdot h_2 = \alpha \cdot y \cdot h^{-1} \cdot h_2 \in (\alpha \cdot y)H$, so $(\alpha \cdot y)H = xH$. This means $g^{-1}(\{xH\}) = (\alpha^{-1} \cdot x)H$, and thus
        $$G / H = \{g^{-1}(\{xH\}): xH \in G / H\}.$$
        By the corollary, $g$ induces a homeomorphism $F_\alpha: G / H \rightarrow G / H$ satisfying $g = F_\alpha \circ p$, or $F_\alpha(xH) = (\alpha \cdot x)H$. To show that $G / H$ is homogeneous, for any $xH, yH \in G / H$ we have $F_{y \cdot x^{-1}}(xH) = yH$. $\Box$

        \item Show that if $H$ is closed in $G$, then one-point sets are closed in $G / H$.

        {\bf SOLUTION.} Suppose $H$ is closed in $G$. Then since $f_x: G \rightarrow G$ is a homeomorphism, $f_x(H) = xH = p^{-1}(\{xH\})$ is closed in $G$, and since $p$ is a quotient map, this means $\{xH\}$ is closed in $G / H$. $\Box$

        \item Show that the quotient map $p: G \rightarrow G / H$ is open.

        {\bf SOLUTION.} If $U$ is open in $G$ then for any $h \in H$, $g_h(U)$ is open in $G$ as $g_h$ is a homeomorphism. Then
        $$p(U) = \bigcup_{x \in U} \{xH\} = \bigcup_{h \in H} g_h(U)$$
        is open in $G / H$. $\Box$
        
        \item Show that if $H$ is closed in $G$ and is a normal subgroup, then $G / H$ is a topological group.

        {\bf SOLUTION.} Suppose $H$ is closed in $G$ and a normal subgroup of $G$. Then $G / H$ is a group as the operation $\cdot: G / H \times G / H \rightarrow G / H$ defined by $(xH) \cdot (yH) = (x \cdot y)H$ is well defined for normal $H$, is associative, has identity element $eH$, and has inverses $(xH)^{-1} = x^{-1}H$. By (b), $G / H$ is moreover $T_1$.
        
        $p: G \rightarrow G / H$ is open by (c), so $p \times p: G \times G \rightarrow G/H \times G/H$ is open, and thus is a quotient map.
        
        Let $f: G \times G \rightarrow G$ be defined by $x \times y \mapsto x \cdot y^{-1}$; since $G$ is a topological group, $f$ is a continuous surjection. Then $p \circ f: G \times G \rightarrow G / H$ defined by $x \times y \mapsto (x \cdot y^{-1})H$ is a composition of continuous surjections. For any $xH \times yH \in G/H \times G/H$ and $w \times z \in (p \times p)^{-1}(\{xH \times yH\})$, we have $wH = xH$ and $zH = yH$, so
        $$(p \circ f)(w \times z) = (w \cdot z^{-1})H = (wH) \cdot (zH)^{-1} = (xH) \cdot (yH)^{-1} = (x \cdot y^{-1})H,$$
        so $p \circ f$ is constant on $(p \times p)^{-1}(\{xH \times yH\})$. Thus $p \circ f$ induces a continuous map $h: G / H \times G / H \rightarrow G / H$ such that $h \circ (p \times p) = p \circ f$, meaning $h(xH \times yH) = (x \cdot y^{-1})H$ for all $x, y \in G$. Then by Exercise 1 and the continuity of $h$, $G / H$ is a topological group. $\Box$
    \end{enumerate}

    \item $\ints$ is a normal subgroup of $(\reals, +)$. What is the quotient $\reals / \ints$?

    {\bf SOLUTION.} Let $p: \reals \rightarrow \reals / \ints$ be the quotient map and let $g: \reals \rightarrow S^1$ be defined by $g(t) = e^{2\pi i t}$. $g$ is a quotient map by $\varepsilon$-$\delta$ arguments. Moreover 
    $$\reals / \ints = \{\{xn : n \in \ints\}: x \in \reals\} = \{\{x \in \reals: e^{2\pi i x} = z\}: z \in S^1 \} = \{g^{-1}(\{z\}): z \in S^1\}.$$
    
    Thus by the corollary, $g$ induces a homeomorphism $f: \reals / \ints \rightarrow S^1$, meaning $\reals / \ints$ may be viewed as the topological group $(S^1, \cdot)$. $\Box$

    \item If $A, B \subseteq G$ let $A \cdot B = \{a \cdot b: a \in A, b \in B\}$. Let $A^{-1} = \{a^{-1}: a \in A\}$.
    \begin{enumerate}
        \item A neighbourhood $V$ of $e$ is symmetric if $V = V^{-1}$. If $U$ is a neighbourhood of $e$, show that there exists a symmetric neighbourhood $V$ of $e$ such that $V \cdot V \subseteq U$.

        {\bf SOLUTION.} Given a neighbourhood $U$ of $e$, there exists a neighbourhood $W$ of $e$ such that $W \cdot W \subseteq U$ by continuity of $\cdot: G \times G \rightarrow G$. By Exercise 1, there exists a neighbourhood $Z$ of $e$ such that $Z \cdot Z^{-1} \subseteq W$ by continuity of $x \times y \mapsto x\cdot y^{-1}$. $Z \cdot Z^{-1}$ is clearly a symmetric neighbourhood of $e$ such that $(Z \cdot Z^{-1}) \cdot (Z \cdot Z^{-1}) \subseteq W \subseteq U$. $\Box$

        \item Show that $G$ is Hausdorff; in fact show that if $x \neq y$ then there exists a neighbourhood $V$ of $e$ such that $V \cdot x$ and $V \cdot y$ are disjoint.

        {\bf SOLUTION.} Given $x\neq y \in G$, $\{x \cdot y^{-1}\}$ is closed in $G$ by the $T_1$ axiom. Then $G - \{x \cdot y^{-1}\}$ is open. By (a), there exists a symmetric neighbourhood $V$ of $e$ such that $V\cdot V \subseteq G - \{x \cdot y^{-1}\}$. Suppose $z \in (V \cdot x) \cap (V \cdot y)$. Then $z = v \cdot x = w \cdot y$ for some $v, w \in V$. This means $x \cdot y^{-1} = v^{-1} \cdot w \in V^{-1} \cdot V = V \cdot V \subseteq G - \{x \cdot y^{-1}\}$, a contradiction. Hence $V \cdot x, V \cdot y$ are disjoint, and thus $G$ is Hausdorff. $\Box$

        \item Show that $G$ satisfies the regularity axiom: Given a closed set $A$ and $x \notin A$, there exist disjoint open sets containing $A$ and $x$, respectively.

        {\bf SOLUTION.} Given $A$ closed in $G$ and $x \in G - A$, $g_{x^{-1}}(A) = A \cdot x^{-1}$ is closed in $G$, where $g_\alpha$ is defined in Exercise 4. Since $G - A \cdot x^{-1}$ is open in $G$, there exists a symmetric neighbourhood $V$ of $e$ such that $V \cdot V \subseteq G - A \cdot x^{-1}$. Suppose $z \in (V \cdot A) \cap (V \cdot x)$ and write $z = v \cdot a = w \cdot x$ for some $v, w \in V, a \in A$. Since $a \cdot x^{-1} = v^{-1} \cdot w$, $A \cdot x^{-1}$ intersects $V \cdot V$, a contradiction. Hence $V \cdot A, V \cdot x$ are disjoint open sets containing $A, x$, respectively. $\Box$

        \item Let $H$ be a subgroup of $G$ that is closed in $G$; let $p: G \rightarrow G / H$ be the quotient map. Show that $G / H$ satisfies the regularity axiom.

        {\bf SOLUTION.} Given $A$ be closed in $G / H$ and $xH \in G / H - A$, $p^{-1}(A)$ is closed and saturated in $G$, satisfying $p(p^{-1}(A)) = A$ since $p$ is surjective and $x \notin p^{-1}(A)$. By (c), there exists a symmetric neighbourhood $V$ of $e$ such that $V \cdot p^{-1}(A)$ and $V \cdot x$ are disjoint open sets containing $p^{-1}(A)$ and $x$, respectively. Since $p$ is an open map, $p(V \cdot p^{-1}(A))$ and $p(V \cdot x)$ are open in $G / H$ and contain $p(p^{-1}(A)) = A$ and $p(x) = xH$, respectively.

        Suppose $z \in p(V \cdot p^{-1}(A)) \cap p(V \cdot x)$. Then there exist $v, w \in V$, $a \in p^{-1}(A)$ such that
        $$z = (v \cdot a)H = (w \cdot x)H,$$
        and thus for some $h \in H$,
        $$w \cdot x = v \cdot a \cdot h.$$
        Then $p(a \cdot h) = (a \cdot h)H = aH \in A$, meaning $a \cdot h \in p^{-1}(A)$. This implies $w \cdot x = v \cdot (a \cdot h) \in (V \cdot x) \cap (V \cdot p^{-1}(A))$, a contradiction. Hence $p(V \cdot p^{-1}(A))$ and $p(V \cdot x)$ are disjoint open sets in $G / H$ containing $A$ and $xH$, respectively, so $G / H$ satisfies the regularity axiom. $\Box$
    \end{enumerate}
\end{enumerate}
\end{document}