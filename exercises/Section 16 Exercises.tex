\documentclass{article}
\usepackage{h}

\title{Topology by James Munkres -- Section 16 Exercises}
\author{William Gao}
\date{Summer 2024}

\begin{document}
\maketitle
\begin{enumerate}
    \item Show that if $Y$ is a subspace of $X$ and $A$ is a subset of $Y$, then the topology $A$ inherits as a subspace of $Y$ is the same as the topology $A$ inherits as a subspace of $X$.

    {\bf SOLUTION.} Let $\topo_X, \topo_Y$ be the subspace topologies of $A$ inherited from $X, Y$, respectively. Consider any $V \cap A \in \topo_Y$, where $V$ is open in $Y$. This means $V = U \cap Y$ for some $U \in \topo_X$. Then
    $$V \cap A = (U \cap Y) \cap A = U \cap (Y \cap A) = U \cap A \in \topo_X,$$
    showing that $\topo_Y \subseteq \topo_X$. Consider any $U \cap A \in \topo_X$, where $U$ is open in $X$. $U \cap Y$ is open in $Y$, and
    $$U \cap A = U \cap (Y \cap A) = (U \cap Y) \cap A \in \topo_Y,$$
    showing $\topo_X \subseteq \topo_Y$. Thus $\topo_X = \topo_Y$. $\Box$

    \item If $\topo, \topo'$ are topologies on $X$ and $\topo'$ is strictly finer than $\topo$, what can you say about the corresponding subspace topologies on the subset $Y$ of $X$?

    {\bf SOLUTION.} The subspace topologies are given by
    $$\topo_Y = \{U \cap Y: U \in \topo\} \text{ and } \topo'_Y = \{V \cap Y: V \in \topo'\}.$$
    Clearly if $U \cap Y \in \topo_Y$ then $U \in \topo$, so $U \in \topo'$, and thus $U \cap Y \in \topo'_Y$. However, the containment is not necessarily strict; we show this by an example. The discrete topology on $\reals$ is strictly finer than the standard topology; but $\ints^+ \subseteq \reals$ inherits the discrete topology from both. Clearly, the finer relation is no longer strict. $\Box$

    \item Consider the set $Y = [-1, 1]$ as a subspace of $\reals$. Which of the following sets are open in $Y$? Which are open in $\reals$?
    \begin{align*}
        A &= \{x: \frac12 < |x| < 1\}, \\
        B &= \{x: \frac12 < |x| \leq 1\}, \\
        C &= \{x: \frac12 \leq |x| < 1\}, \\
        D &= \{x: \frac12 \leq |x| \leq 1\}, \\
        E &= \{x: 0 < |x| < 1 \text{ and } \frac1x \notin \ints^+ \}.
    \end{align*}

    {\bf SOLUTION.} $A = (-1, -\frac12) \cup (\frac12, 1)$ is the union of two open intervals, hence it is open in $\reals.$ Since $A \cap Y = A$, $A$ is open in $Y$.

    $B = [-1, -\frac12) \cup (\frac12, 1]$ is not open in $\reals$, but it is open in $Y$ because $B' = (-2, -\frac12) \cup (\frac12, 2)$ is open in $\reals$ and $B = B' \cap Y$.

    $C = (-1, -\frac12] \cup [\frac12, 1)$ is not open in $\reals$, and not open in $Y$.

    $D = [-1, -\frac12] \cup [\frac12, 1]$ is open in neither $\reals$ nor $Y$.

    $E = (-1, 0) \cup (0, 1) - K = (-1, 0) \cup \bigcup_{n \in \nats} (\frac{1}{n+1}, \frac1n)$ is open in $\reals$, and $E \cap Y = E$, so it is open in $Y$. $\Box$
    
    \item A map $f: X \rightarrow Y$ is {\it open} if for every $U$ open in $X$, $f(U)$ is open in $Y$. Show that $\pi_1: X \times Y \rightarrow X$ and $\pi_2 : X \times Y \rightarrow Y$ are open maps.

    {\bf SOLUTION.} Suppose $U \times V$ is open in $X \times Y$. Then for any $(x, y) \in U \times V,$ there exists a basis element $W \times Z$ such that $(x, y) \in W \times Z \subseteq U \times V$. This means $x \in W \subseteq U$ and $y \in Z \subseteq V$. By definition, $W$ is open in $X$ and $Z$ is open in $Y$. Since
    $$x \in \pi_1(W \times Z) \subseteq \pi_1(U \times V) \text{ and } y \in \pi_2(W \times Z) \subseteq \pi_2(U \times V),$$
    $\pi_1(U \times V)$ and $\pi_2(U \times V)$ are open in $X$ and $Y$, respectively. Thus $\pi_1, \pi_2$ are open maps. $\Box$

    \item Let $\topo, \topo'$ be topologies on $X$ and $\mathcal{U}, \mathcal{U}'$ topologies on $Y$, where $X$ and $Y$ are nonempty.
    \begin{enumerate}
        \item Suppose that $\topo' \supseteq \topo$ and $\mathcal{U}' \supseteq \mathcal{U}$. Show that the product topology from $\topo'$ and $\mathcal{U}'$ is finer than the product topology from $\topo$ and $\mathcal{U}$.

        {\bf SOLUTION.} Let $\basis, \basis', \cee, \cee'$ be bases for $\topo, \topo', \mathcal{U}, \mathcal{U}'$, respectively. By the finer relations, for every $B \in \basis$ and every $x \in B$, there exists $B' \in \basis'$ such that $x \in B' \subseteq B$, and for every $C \in \cee$ and every $y \in C$, there exists $C' \in \cee'$ such that $y \in C' \subseteq C$.

        Consider any basis element $U \times V$ for $\topo \times \mathcal{U}$ and any $(x, y) \in U \times V$. By definition, $U \in \basis$ and $V \in \cee$. Let $W \in \basis', Z \in \cee'$ be such that $x \in W \subseteq U, y \in Z \subseteq V$. Then $(x, y) \in W \times Z \subseteq U \times V$, and since $W \times Z$ is a basis element of the product topology on $X' \times Y'$, the product topology on $X' \times Y'$ is finer than that on $X \times Y.$ $\Box$ 
        
        \item Does the converse hold? 

        {\bf SOLUTION.} Suppose the product topology from $\topo'$ and $\mathcal{U}'$ is finer than that from $\topo$ and $\mathcal{U}$. Then for every basis element $U \times V$ for $\topo \times \mathcal{U}$ and every $(x, y) \in U \times V$, there exists a basis element $U' \times V'$ satisfying $(x, y) \in U' \times V' \subseteq U \times V$. This means $x \in U' \subseteq U$ and $y \in V' \subseteq V$. Since any basis elements $U \in \topo$ and $V \in \mathcal{U}$ and can be found in a basis element $U \times V$ for $\topo \times \mathcal{U}$, we see that $\topo'$ is finer than $\topo$ and $\mathcal{U}'$ is finer than $\mathcal{U}.$ $\Box$ 

        \end{enumerate}

        \item Show that the countable collection $\{(a, b) \times (c, d) : a< b, c<d, a, b, c, d \in \rats\}$ is a basis for $\reals^2$. 

        {\bf SOLUTION.} By exercise 7 in problem set 1, $\cee = \{(a, b) : a< b \text{ and } a, b \in \rats\}$ is a basis for the standard topology on $\reals.$ Thus by Theorem 2.10, $\cee$ is a basis for $\reals \times \reals$. $\Box$ 

        \item Let $X$ be an ordered set and $Y \subsetneq X$ convex in $X$. Does it follow that $Y$ is an interval or a ray in $X$?

        {\bf SOLUTION.} No. As a counterexample, consider $X =\reals^2$ and $Y = \{x \times y: -1 \leq x \leq 1\}$. To see that $Y$ is convex, suppose $a \times b, c \times d \in Y$, meaning $-1 \leq a, c \leq 1$. Then suppose $x \times y \in (a \times b, c \times d)$. By the dictionary order, $a \leq x \leq c$, and thus $-1 \leq x \leq 1$, so $x \times y \in L$. This shows that $(a \times b, c \times d) \subseteq Y$, meaning $Y$ is convex. $\Box$ 

        \item If $L$ is a straight line in the plane, describe the topology $L$ inherits as a subspace of $\reals_\ell \times \reals$ and as a subspace of $\reals_\ell \times \reals_\ell$. 

        {\bf SOLUTION.} Let $L$ be a straight line. Basis elements of $\reals_\ell \times \reals$ are in the form $[a, b) \times (c, d)$, which may be represented by a rectangle in the plane with all sides closed except for the left. The topology $L$ inherits as a subspace of $\reals_\ell \times \reals$ is generated by
        $$\{ L \cap [a, b) \times (c, d)\},$$
        hence it is the lower limit topology on $L$.

        Basis elements of $\reals_\ell \times \reals_\ell$ are in the form $[a, b) \times [c, d)$, or a rectangle with left and bottom sides closed. If $L$ is horizontal or vertical, then its intersection with the basis is the set of all intervals $[a, b)$ or $[c, d)$, respectively. This is a basis for the lower limit topology on $L$.

        If $L$ has positive slope, then its intersection with the basis is a set of intervals $[a, b)$, and thus generates the lower limit topology.

        If $L$ has negative slope, we see that for every $(x, y) \in L$, the intersection of the basis element $[x, y) \times [x+1, y+1)$ of $\reals_\ell \times \reals_\ell$ with $L$ is $\{(x, y)\}$. Thus every point on $L$ is a basis element, meaning $L$ inherits the discrete topology. $\Box$ 

        \item Show that the dictionary order topology on $\reals^2$ is the same as the product topology $\reals_d \times \reals$. Compare this topology with the standard topology on $\reals^2$. 

        {\bf SOLUTION.} Let $\basis = \{\{x\}: x \in \reals\}, \cee = \{(a, b): a, b \in \reals\}$ be bases for $\reals_d, \reals$, respectively. Then $\basis \times \cee$ is a basis for the product topology on $\reals_d \times \reals$. 
        
        Consider an arbitrary element $\{x\} \times (a, b) \in \basis \times \cee$. $(x \times a, x \times b)$ is an open interval in the dictionary order topology that is equal to $\{x\} \times (a, b)$, hence $\reals_d \times \reals$ is coarser than the dictionary order topology.
        
        Conversely, consider an arbitrary basis element $(a \times b, c \times d)$ in the dictionary order topology and any $x \times y \in (a \times b, c \times d)$. There exists an open interval $(e, f)$ such that $y \in (e, f) \subseteq (b, c)$. Then $\{x\} \times (e, f)$ is a basis element of $\reals_d \times \reals$ containing $x \times y$ that is contained by $(a \times b, c \times d)$, so $\reals_d \times \reals$ is finer than the dictionary order topology.

        $\mathcal{D} = \{(a, b) \times (c, d): a, b, c,d\in \reals\}$ is a basis for $\reals^2$. We see that given a basis element $\{x\} \times (c, d) \in \reals_d \times \reals$ containing $x \times y$, there is no $(a, b) \times (c, d) \in \reals^2$ that contains $x \times y$ and is contained by $\{x\} \times (c, d)$, since any open interval containing $x$ is a strict superset of $\{x\}$. This means the standard topology is not finer than the product topology. Conversely, given $(a, b) \times (c,d) \in \reals^2$ containing $x \times y$, $\{x\} \times (c, d)$ is a basis element of $\reals_d \times \reals$ such that $x \times y \in \{x\} \times (c, d) \subseteq (a, b) \times (c, d)$. Thus the product topology is strictly finer than the standard topology. $\Box$ 

        \item Let $I = [0, 1]$. Compare the product topology on $I \times I$, the dictionary order topology on $I \times I$, and the topology $I \times I$ inherits as a subspace of $\reals \times \reals$ in the dictionary order topology. 

        {\bf SOLUTION.} Let $\topo_1, \topo_2, \topo_3$ denote these three topologies, respectively. We know $\topo_1$ is the subspace topology on $I^2 \subseteq \reals^2$ and $\topo_3$ is the subspace topology of $\reals_d \times \reals$, hence these topologies are generated by
        \begin{align*}
            \basis_1 &= \{ ((a, b) \cap [0, 1]) \times ((c, d) \cap [0, 1]): a,b,c,d\in \reals\}, \\
            \basis_2 &= \{(a \times b, c \times d)\} \cup\{[0 \times 0, a\times b)\} \cup \{(a \times b, 1 \times 1]\}, \\
            \basis_3 &= \{ \{x\} \times ((a, b) \cap [0, 1]) : x \in [0, 1], a,b \in \reals\}.
        \end{align*}
        Consider an arbitrary basis element $(a, b) \cap [0, 1] \times (c, d) \cap [0, 1] \in \basis_1$ containing $x \times y$. There is no basis element in $\basis_2$ containing $x$ that is contained by $(a, b) \cap [0, 1] \times (c, d) \cap [0, 1]$ if $(c, d) \subsetneq [0, 1].$ This means $\topo_2$ is not finer than $\topo_1.$ However, $\{x \} \times (c, d) \cap [0, 1] \in \basis_3$ satisfies $x \times y \in \{x \} \times (c, d) \cap [0, 1] \subseteq (a, b) \cap [0, 1] \times (c, d) \cap [0, 1],$ hence $\topo_3$ is finer than $\topo_1$.

        Consider an arbitrary basis element $(a \times b, c \times d) \in \basis_2$ containing $x \times y$. Let $(p, q) \subseteq \reals$ be such that $x \in (p, q) \subseteq (a, c)$ and let $(r, s) \subseteq \reals$ be such that $y \in (r, s) \subseteq (b, d)$. Then $(p, q) \cap [0, 1] \times (r, s) \cap [0, 1] \in \basis_1$ contains $x \times y$ and is contained by $(a \times b, c \times d)$. Identically for basis elements in the form $[0 \times 0, a\times b)$ and $(a \times b, 1 \times 1]$, we see that $\topo_1$ is strictly finer than $\topo_2$. It follows that $\topo_3$ is finer than $\topo_2$.

        Consider an arbitrary basis element $\{x \} \times (a, b) \cap [0, 1] \in \basis_3$ containing $x \times y$. Since any open interval around $x$ is a strict superset of $\{x\}$, there is no basis element in $\basis_1$ containing $x\times y$ that is contained by $\{x\} \times (a, b) \cap [0, 1]$, meaning $\topo_1$ is not finer than $\topo_3$. Therefore $\topo_3 \supsetneq \topo_1 \supsetneq \topo_2$. $\Box$
\end{enumerate}
\end{document}