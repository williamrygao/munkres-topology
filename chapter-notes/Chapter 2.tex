\section{Topological Spaces and Continuous Functions}

\subsection{Topological Spaces}
\begin{definition}\label{2.1}
    A {\it topology} on a set $X$ is a collection $\topo$ of subsets of $X$ with the following properties:
    \begin{enumerate}
        \item[(1)] $\varnothing, X \in \topo$.
        \item[(2)] the union of the elements in any subcollection of $\topo$ is in $\topo$.
        \item[(3)] the intersection of the elements of any finite subcollection of $\topo$ is in $\topo$.
    \end{enumerate}
    A {\it topological space} is an ordered pair $(X, \topo)$ consisting of a set $X$ and a topology $\topo$ on $X$.

    A $U \subseteq X$ is an {\it open set} of $X$ if $U \in \topo$.
\end{definition}

A few general examples of topologies:
\begin{itemize}
    \item The {\it discrete topology} of all subsets of $X$. 
    \item The {\it indiscrete (or trivial) topology} of $X$ and $\varnothing$.
    \item The {\it finite complement topology} $\topo_f$, the collection of all $U \subseteq X$ such that $X - U$ is finite or all of $X$. We see that for an indexed family $\{U_\alpha\}$ of nonempty elements in $\topo_f$,
    $$X - \bigcup U_\alpha = \bigcap (X - U_\alpha),$$
    which is finite since each $X - U_\alpha$ is finite. For a finite list $U_1, \cdots, U_n$ of nonempty elements in $\topo_f$,
    $$X - \bigcup_{i=1}^n U_i = \bigcup_{i=1}^n (X- U_i),$$
    which is finite.
    \item Similarly, the {\it countable complement topology} $\topo_c$, the collection of all $U \subseteq X$ such that $X - U$ is either countable or all of $X$.
\end{itemize}

Given two topologies $\topo, \topo'$ on a set $X$, we say $\topo$ is {\it comparable} with $\topo'$ if $\topo \subseteq \topo'$ or $\topo \supseteq \topo'$. In the former case, $\topo$ is {\it coarser} than $\topo'$; in the latter case, $\topo$ is {\it finer} than $\topo'$. Moreover, we say strictly if the containment is strict.

\subsection{Basis for a Topology}
In general, describing a topology by the entire collection $\topo$ is difficult. It is possible to characterize a topology by a subcollection of $\topo$, called a basis.
\begin{definition}\label{2.2}
    Given a set $X$, a {\it basis} for a topology on $X$ is a collection $\mathcal{B}$ of subsets of $X$, called basis elements, such that
    \begin{enumerate}
        \item[(1)] for each $x \in X,$ there is at least one basis element containing $x$.
        \item[(2)] if $B_1, B_2 \in \mathcal{B}$ and $x \in B_1 \cap B_2$, then there exists $B_3 \in \mathcal{B}$ such that $x \in B_3 \subseteq B_1 \cap B_2$.
    \end{enumerate}
    From this, we define the topology $\topo$ generated by $\mathcal{B}$: $U \subseteq X$ is open ($\in \topo$) if for each $x \in U,$ there exists a $B \in \mathcal{B}$ such that $x \in B \subseteq U$. In particular, every basis element is in $\topo.$
\end{definition}
It is not immediately clear that the collection $\topo$ generated by a basis $\mathcal{B}$ must necessarily be a topology on $X$. We readily see that $\varnothing$ is open, and so is $X$, since for each $x \in X$ there is some $B \in \mathcal{B}$ such that $x \in B \subseteq X$.

Suppose $\{U_\alpha\}_{\alpha \in J}$ is an indexed family of elements of $\topo$. Given $x \in U = \bigcup_{\alpha \in J}U_\alpha$, $x \in U_\alpha$ for some $\alpha \in J$. Since $U_\alpha \in \topo$, $x \in B \subseteq U_\alpha$ for some $B \in \mathcal{B}$. Since $U_\alpha \subseteq U,$ $U$ is open.

Now let $U_1, U_2 \in \topo$; given $x\in U_1 \cap U_2$, let $B_1, B_2 \in \mathcal{B}$ be such that $x \in B_1 \subseteq U_1$ and $x \in B_2 \subseteq U_2$. By the second property of $\mathcal{B}$, there exists $B_3 \in \mathcal{B}$ such that $x \in B_3 \subseteq B_1 \cap B_2$. Hence $U_1 \cap U_2 \in \topo$, by definition. By induction, any finite intersection $U_1 \cap \cdots \cap U_n$ of elements of $\topo$ is in $\topo$. $\Box$

Generating a topology from \hyperref[2.2]{2.2} is often quite strenuous; the following provides a much more straightforward procedure.
\begin{lemma}\label{2.3}
    Let $X$ be a set and $\mathcal{B}$ a basis for a topology $\topo$ on $X$. Then $\topo$ is the collection of all unions of elements in $\mathcal{B}$.
\end{lemma}
{\it Proof.} Any element in $\mathcal{B}$ is also an element of $\topo$, by definition. Moreover, their union is in $\topo$. Conversely, given $U \in \topo,$ for each $x \in U$ there exists an element $B_x \in \mathcal{B}$ such that $x \in B_x \subseteq U$. Then $U = \bigcup_{x\in U} B_x$, so every $U \in \topo$ is a union of elements in $\mathcal{B}$. $\Box$

Now that we have a convenient method for obtaining a topology from a given basis, we introduce a method for obtaining a basis for a given topology.
\begin{lemma}\label{2.4}
    Let $X$ be a set with a topology $\topo$. Suppose $\mathcal{C} \subseteq \topo$ is such that for every $U \in \topo$ and $x \in U$, there exists $C \in \mathcal{C}$ such that $x \in C \subseteq U$. Then $\mathcal{C}$ is a basis for $\topo$.
\end{lemma}
{\it Proof.} We must show that $\mathcal{C}$ is a basis. For the first condition, given $x \in X$, $X \in \topo$ implies there $C \in \mathcal{C}$ such that $x \in C \subseteq X$.

For the second condition, suppose $x \in C_1\cap C_2$, where $C_1, C_2 \in \mathcal{C}$. $C_1, C_2 \in \topo$, hence $C_1 \cap C_2 \in \topo$. Then there exists $C_3 \in \mathcal{C}$ such that $x \in C_3 \subseteq C_1 \cap C_2$.

Now we must show that the topology $\topo'$ generated by $\mathcal{C}$ is in fact $\topo$. Firstly, if $U \in \topo$ and $x \in U$, then there exists $C \in \mathcal{C}$ such that $x \in C \subseteq U$, so $U \in \topo'$. Conversely, if $W \in \topo'$, then $W$ is a union of elements in $\mathcal{C}$ by \hyperref[2.3]{2.3}. Since every element in $\mathcal{C}$ is in $\topo$, $W \in \topo.$ $\Box$

The first powerful use of bases is the following criterion for determining whether one topology is finer than another, in terms of their bases.
\begin{lemma}\label{2.5}
    Let $X$ be a set with topologies $\topo, \topo'$ generated by $\mathcal{B}, \mathcal{B}'$, respectively. Then $\topo' \supseteq \topo$ if and only if for each $x \in X$ and each $B \in \mathcal{B}$ containing $x$, there exists $B' \in \mathcal{B}'$ such that $x \in B' \subseteq B$.
\end{lemma}
{\it Proof.} $(\impliedby)$. Given $U \in \topo$ and $x \in U,$ there exists $B \in \mathcal{B}$ such that $x \in B \subseteq U$. Then there exists $B' \in \mathcal{B}'$ such that $x \in B' \subseteq B \subseteq U$, so $U \in \topo'$. 

$(\implies)$. Suppose $x \in X$ and $B \in \mathcal{B}$ are such that $x \in B$. Then $B \in \topo \subseteq \topo'$. Since $\topo'$ is generated by $\mathcal{B}',$ there exists $B' \in \mathcal{B}'$ with $x \in B' \subseteq B$. $\Box$

So far we have only considered topologies on arbitrary sets. We now define three topologies on $\reals$.

\begin{itemize}
    \item Let $\mathcal{B} = \{(a, b) : a, b \in \reals\}$. The topology generated by $\mathcal{B}$ is the {\it standard topology} on $\reals.$
    \item Let $\mathcal{B}' = \{[a, b) : a, b \in \reals\}$. The topology generated by $\mathcal{B}'$ is the {\it lower-limit topology} on $\reals,$ denoted $\reals_\ell$.
    \item Let $K = \left\{\frac{1}{n}: n \in \ints^+\right\}$. Let $\mathcal{B}'' = \{(a, b) : a, b \in \reals\} \cup \{(a, b) - K: (a, b) \in \reals\}$. The topology generated by $\mathcal{B}''$ is the {\it K-topology} on $\reals,$ denoted $\reals_K$.
\end{itemize}
\vspace{10pt}

\begin{center}
    {\it ``The K-topology serves no purpose other than to confuse you.}

    -- The late great Duke Ellington
\end{center}
\vspace{10pt}

\begin{lemma}\label{2.6}
    $\reals_\ell$ and $\reals_K$ are strictly finer than the standard topology, but not comparable with one another.
\end{lemma}
{\it Proof.} Let $\topo, \topo_\ell, \topo_K$ denote the topologies of $\reals, \reals_\ell, \reals_K$, respectively. Given $(a, b) \in \topo$ and $x \in (a, b)$, $[x, b) \in \topo_\ell$ satisfies $x \in [x, b)$ and $[x, b) \subseteq (a, b)$. Conversely, given $[x, d) \in \topo_\ell$ and $x \in [x, d)$, there is no open interval that contains $x$ and is contained in $[x, d)$, hence $\topo_\ell \supsetneq \topo.$

Given $(a, b) \in \topo$, we also have $(a, b) \in \topo_K$, so clearly $\topo_K \supseteq \topo$. Conversely, consider $(-1, 1) - K \in \topo_K$ and $0 \in (-1, 1)- K$. There is no open interval that contains $0$ and is contained in $(-1, 1) - K$, since any open interval around $0$ contains $\frac1n$ for all $n$ greater than some $N$ as $\lim_{n \rightarrow \infty} \frac1n = 0$. Thus $\topo_K \supsetneq \topo$.

To show that $\topo_\ell$ and $\topo_K$ are not comparable, we remark that given $[a, b) \in \topo_\ell$ and $a \in [a, b)$, there is no basis element for $\topo_K$ that contains $a$ and is contained by $[a, b)$. Conversely, given $(-1, 1) - K \in \topo_K$ and $0 \in (-1, 1) - K$, there is no $[a, b) \in \topo_\ell$ that contains $0$ and is contained by $(-1, 1) - K$. $\Box$

A basis generates a topology by arbitrary unions of its elements. We may also consider a collection of subsets of $X$ that generates a topology by arbitrary unions of finite intersections.
\begin{definition}\label{2.7}
    A {\it subbasis} $\mathcal{S}$ for a topology on $X$ is a collection of subsets of $X$ whose union equals $X$. The topology generated by $\mathcal{S}$ is defined as the collection $\topo$ of all unions of finite intersections of elements of $\mathcal{S}.$
\end{definition}
To show that $\topo$ is in fact a topology, it suffices to show that the collection $\mathcal{B}$ of all finite intersections of elements in $\mathcal{S}$ is a basis. Given $x \in X$, there exists $S \in \mathcal{S}$ such that $x \in S$, hence there exists $B \in \mathcal{B}$ such that $x \in B$, satisfying the first condition.

For any $B_1, B_2 \in \mathcal{B}$, $B_1$ and $B_2$ are finite intersections of elements in $\mathcal{S}$, by definition. Then $B_1 \cap B_2$ is also a finite intersection of elements in $\mathcal{S}$, so $B_3 = B_1 \cap B_2 \in \mathcal{B}$, satisfying the second condition.

Since $\mathcal{B}$ is a basis, the collection $\topo$ of all unions of its elements is a topology by \hyperref[2.2]{2.2}.

\subsection{The Order Topology}
Given a simply ordered set $X$, we define a canonical topology on $X$.
\begin{definition}\label{2.8}
    Suppose $X$ is a set, containing more than one element, with a simple order relation. Let $\mathcal{B}$ be defined as the collections of all open intervals $(a, b) \subseteq X$, all intervals of the form $[a_0, b)$, where $a_0$ is the smallest element (if any) of $X$, and all intervals of the form $(a, b_0]$, where $b_0$ is the largest element (if any) of $X$.

    $\mathcal{B}$ is a basis for the {\it order topology} on $X$.
\end{definition}
To verify that $\mathcal{B}$ indeed defines a basis, every $x \in X$ lies in at least one element of $\mathcal{B}$: the smallest in $[a_0, b),$ the largest in $(a, b_0],$ and every other in $(a, b)$. 

Moreover, the intersection of any two open intervals is empty or again open; the intersection of any intervals of the form $[a_0, b)$ is again in this form; and symmetrically for $(a, b_0]$. $(a, b) \cap [a_0, c) = \varnothing$ or $(a, \min \{b, c\})$, $(a, b) \cap (c, b_0] = \varnothing$ or $(\max\{a, c\}, b)$, and $[a_0, b) \cap (a, b_0] = \varnothing$ or $(a, b)$. Thus the second property for a basis is satisfied.

From this definition, we see that the standard topology on $\reals$ is in fact the order topology. On the other hand, the order topology on $\ints^+$ is the discrete topology. Every one-point set $\{n\}$ is a basis element, since it may be written as $[1, 2)$ if $n = 1$ or $(n-1, n+1)$ if $n>1.$

$\reals^2$ has no smallest or largest element, so the dictionary order topology on $\reals^2$ is generated by the collection of all open intervals of the form $(a \times b, c \times d)$ for $a < c$, or $a = c$ and $b < d$.

Contrastingly, $X = \{1, 2\} \times \ints^+$ in the dictionary order has a smallest element $1 \times 1.$ By the dictionary order, we can represent $X$ by
$$1 \times 1, 1 \times 2, \cdots; 2 \times 1, 2 \times 2, \cdots$$
The order topology on $X$ is not the discrete topology, since the one-point set $\{2\times 1\}$ is not open; any basis element containing $1\times 2$ must contain some points in the sequence $\{1 \times n\}$, and thus is not contained in $\{2\times 1\}$.

Given $a \in X$, the following four subsets, called {\it rays}, with be a useful shorthand:
\begin{align*}
    (a, \infty) &= \{x: x> a\}, \\
    (-\infty, a) &= \{x: x< a\}, \\
    [a, \infty) &= \{x: x\geq a\}, \\
    (-\infty, a] &= \{x: x \leq a\}.
\end{align*}
The set of all open rays form a subbasis for the order topology on $X$.

\subsection{The Product Topology}
Given topological spaces $X, Y$, there exists a canonical topology on $X \times Y$.
\begin{definition}\label{2.9}
    Let $\mathcal{B}$ be defined as the collection of all sets of the form $U \times V$, where $U$ is open in $X$ and $V$ is open in $Y$. $\mathcal{B}$ generates the {\it product topology} on $X \times Y$.
\end{definition}
To verify that $\mathcal{B}$ is necessarily a basis, it suffices to check the second condition, since the first is automatic from $X \times Y \in \mathcal{B}$. We see that
$$(U_1 \times V_1) \cap (U_2 \times V_2)= (U_1 \cap U_2) \times (V_1 \cap V_2),$$
and since $U_1, U_2, V_1, V_2$ are open, so are $U_1 \cap U_2$ and $V_1 \cap V_2$. Thus their intersection is also a basis element.
\begin{theorem}\label{2.10}
    Suppose $X$ and $Y$ are topological spaces with bases $\mathcal{B}, \mathcal{C}$, respectively. Then
    $$\mathcal{D} = \{B \times C: B \in \mathcal{B}, C \in \mathcal{C}\}$$
    is a basis for the topology of $X \times Y$.
\end{theorem}
{\it Proof.} Given an open set $W \subseteq X \times Y$ and a point $x \times y \in W$, by definition of the product topology there exists a basis element $U \times V$ such that $x \times y \in U \times V \subseteq W.$ Since $\mathcal{B}, \mathcal{C}$ are bases, there exist $B \in \mathcal{B}, C \in \mathcal{C}$ such that $x \in B \subseteq U$ and $y \in C \subseteq V$; that is, $x \times y \in B \times C \subseteq W$. Thus by \hyperref[2.4]{2.4}, $\mathcal{D}$ is a basis for $X \times Y$. $\Box$

As an example, we may define the standard topology on $\reals^2$ as the product topology of the standard topologies on $\reals$. By \hyperref[2.10]{2.10}, it has as basis the collection of all products of open intervals in $\reals$, which may be visualized as the interiors of rectangles.

To extend this idea to subbases, we must first define a specific class of functions.

\begin{definition}\label{2.11}
    The {\it projections} of $X \times Y$ onto its first and second factors, $\pi_1:X \times Y \rightarrow X$ and $\pi_2: X \times Y \rightarrow Y$, are defined by
    \begin{align*}
        \pi_1(x, y) &= x, \\
        \pi_2(x, y) &= y.
    \end{align*}
\end{definition}
We remark that $\pi_1, \pi_2$ are surjective. Thus if $U \subseteq X$ is open, then $\pi_1^{-1}(U)= U \times Y$ is open, and if $V \subseteq Y$ is open, then $\pi_2^{-1} (V) = X \times V$ is also open. The following theorem relates these observations to subbases.
\begin{theorem}\label{2.12}
    $\mathcal{S} = \{\pi_1^{-1}(U): U \text{ open in } X\} \cup \{\pi_2^{-1}(V): V \text{ open in } Y\}$ is a subbasis for the product topology on $X \times Y$.
\end{theorem}
{\it Proof.} Let $\topo$ denote the product topology on $X \times Y$; let $\topo'$ be the topology generated by $\mathcal{S}$. Since every element in $\mathcal{S}$ is in $\topo$ by the observations above, arbitrary unions of finite intersections in $\mathcal{S}$ belong to $\topo,$ so $\topo' \subseteq \topo$. Conversely, every basis element $U \times V$ of $\topo$ may be written as
$$U \times V = \pi_1^{-1}(U) \cap \pi_2^{-1} (V),$$
and thus is a finite intersection of elements of $\mathcal{S}$. Therefore, $U \times V \in \topo'$, showing that $\topo = \topo'$. $\Box$

\subsection{The Subspace Topology}
In this section, we treat subsets of topological spaces with topologies of their own.
\begin{definition}\label{2.13}
    Suppose $X$ is a set with topology $\topo$. If $Y \subseteq X$, then we define the {\it subspace topology} on $Y$ by
    $$\topo_Y = \{Y \cap U: U \in \topo\}.$$
    We call $Y$ a subspace of $X$.
\end{definition}
To verify that $\topo_Y$ is a topology, we see that since $\varnothing, X \in \topo$,
$$\varnothing = Y \cap \varnothing \in \topo_Y \text{ and } Y = Y \cap X \in \topo_Y.$$
Closure under finite intersections and arbitrary unions follow similarly.
\begin{align*}
    \bigcup_{\alpha \in J} (U_\alpha \cap Y) &= ( \bigcup_{\alpha \in J} U_\alpha ) \cap Y, \\
    \bigcap_{i=1}^n (U_i \cap Y) &= (\bigcap_{i=1}^n (U_i) \cap Y.
\end{align*}
To construct the subspace topology from a basis for $\topo$, we have the following lemma.
\begin{lemma}\label{2.14}
    Suppose $X$ is a set with a topology generated by $\mathcal{B}$. Then
    $$\mathcal{B}_Y = \{B \cap Y: B \in \mathcal{B}\}$$
    is a basis for the subspace topology on $Y$.
\end{lemma}
{\it Proof.} Given an open set $U \subseteq X$ and any $y \in U \cap Y$, let $B \in \mathcal{B}$ be such that $y \in B \subseteq U$. Then $y \in B \cap Y \subseteq U \cap Y$. By \hyperref[2.4]{2.4}, $\mathcal{B}_Y$ is a basis for the subspace topology on $Y.$ $\Box$

Note that a subset $U$ may be open in $Y$ but not $X$; or vice versa. In fact, the following lemma addresses a special situation related to this.
\begin{lemma}\label{2.15}
    Suppose $Y$ is a subspace of $X$. If $U$ is open in $Y$ and $Y$ is open in $X$, then $U$ is open in $X$.
\end{lemma}
{\it Proof.} Since $U$ is open in $Y$, $U = Y \cap V$ for some $V$ open in $X$. Then since $Y$ and $V$ are open in $X$, so is $U = Y \cap V$. $\Box$

We now consider the relation between the subspace topology and the order/product topologies.
\begin{theorem}\label{2.16}
    Suppose $A, B$ are subspaces of $X, Y$, respectively. Then the product topology on $A \times B$ is the subspace topology on $A \times B$.
\end{theorem}
{\it Proof.} Basis elements for the subspace topology on $A \times B$ are given by $(U \times V) \cap (A \times B)$, where $U \times V$ is open in $X \times Y$. Equivalently, $U$ is open in $X$ and $V$ is open in $Y$. Since
$$(U \times V) \cap (A \times B) = (U \cap A) \times (V \cap B),$$
and $U \cap A, V \cap B$ describe any basis elements of the subspace topologies on $A, B$, respectively, $(U \cap A) \times (V \cap B)$ describes any basis elements for the product topology on $A \times B$. Thus the two topologies are the same. $\Box$

This result relating the subspace topology and the product topology is perfectly ideal, but the order topology will be far more subtle. First, we consider some examples.

Consider $X = \reals, Y = [0, 1]$. For $(a, b)$ open in $\reals,$ the subspace topology of $Y$ is generated by the set of intervals in the form
$$(a, b) \cap [0, 1] = \begin{cases}
    (a, b) &\text{if }a, b \in [0, 1] \\
    [0, b] &\text{if only } b \in [0, 1] \\
    (a, 1] &\text{if only } a\in [0, 1] \\
    Y, \varnothing &\text{if }a, b \notin Y.
\end{cases}$$
Moreover, these sets form a basis for the order topology on $[0, 1]$. Thus the subspace topology and the order topology of $[0, 1]$ are the same.

Consider $X = \reals, Y= [0, 1) \cup \{2\}$. In the subspace topology on $Y$, $\{2\}$ is open, as it may be seen as the intersection of $(0, 3)$ with $Y$. However in the order topology on $Y$, $\{2\}$ is not open since any basis element containing $2$ must must be in the form $(a, 2]$, which is not contained in $\{2\}$.

As another anomaly, let $I_0 = [0, 1]$, and consider $X = \reals \times \reals, Y = I \times I$ with the dictionary order. $\{\frac12\} \times (\frac12, 1]$ is open in the subspace topology of $I \times I$, but not in the order topology, since there is no basis element containing $(\frac12, 1)$ that is contained in $\{\frac12\} \times (\frac12, 1]$.

The following definition will lead us to a theorem summarizing the discussion of the order and subspace topologies.
\begin{definition}\label{2.17}
    Given an ordered set $X$, $Y \subseteq X$ is {\it convex} in $X$ if for each $a, b \in Y$ with $a< b$, $(a, b) \subseteq Y$.    
\end{definition}
In particular, intervals and rays are convex.

\begin{theorem}\label{2.18}
    Suppose $X$ is an ordered set with the order topology, and $Y$ a convex subset of $X$. Then the order topology on $Y$ is the subspace topology of $Y$.
\end{theorem}
{\it Proof.} We use the fact that sets in the form $(a, \infty) \cap Y$ and $(-\infty, a) \cap Y$ form a subbasis for the subspace topology on $Y$ to show that the order topology is finer than the subspace topology. We have
$$(a, \infty) \cap Y = \{x : x \in Y, x > a\},$$
which is an open ray in $Y$ if $a \in Y.$ Otherwise $a \notin Y$, then since $Y$ is convex, $a$ is either a lower bound or an upper bound for $Y$. In the former case, $(a, \infty) \cap Y = Y$; in the latter case, $(a, \infty) \cap Y = \varnothing$. Similarly, $(-\infty, a) \cap Y$ is either an open ray in $Y$, empty, or $Y$ itself. Since $(a, \infty) \cap Y$ and $(-\infty, a) \cap Y$ form a subbasis for the subspace topology on $Y$, and are both open in the order topology on $Y$, the order topology is finer than the subspace topology.

Conversely, the collection of open rays in $Y$ is a subbasis for the order topology on $Y.$ Since any open ray in $Y$ may be written as the intersection of an open ray in $X$ with $Y$, it is open in the subspace topology on $Y$. Thus the subspace topology is finer than the order topology, completing the equivalence. $\Box$

From this point on, we will assume $Y \subseteq X$ is given the subspace topology unless otherwise stated.

\subsection{Closed Sets and Limit Points}
\begin{definition}\label{2.19}
    A subset $A$ of a topological space $X$ is {\it closed} if its complement $X -A$ is open.
\end{definition}
While closed is a grammatical opposite to open, a set can be open, closed, both, or neither, as demonstrated by the following examples.

\begin{itemize}
    \item Consider $A = [a, b] \subseteq \reals$. $A$ is closed because its complement $\reals - [a, b] = (-\infty, a) \cup (b, \infty)$ is clearly open. Similarly, a closed ray $[a, \infty)$ is closed. However, half-closed intervals $[a, b)$ are neither open nor closed.
    \item Consider $A = \{x \times y: x\geq 0, y \geq 0\} \subseteq \reals^2$. $A$ is closed because its complement $(-\infty, 0) \times \reals \cup \reals \times (-\infty, 0)$ is a union of products of open sets in $\reals$ and therefore open in $\reals^2$.
    \item In the finite complement topology on a set $X$, the closed sets consist of $X$ and all finite subsets of $X$.
    \item In the discrete topology on a set $X$, every set is open, hence every set is closed.
    \item Consider $Y = [0, 1] \cup (2, 3)$ in the subspace topology inherited from $\reals$. Both $[0, 1]$ and $(2, 3)$ are open in $Y$, and since they are each other's complements in $Y$, they are both closed in $Y$.
\end{itemize}

The collection of closed subsets of a topological space has some nice properties almost identical to those of open subsets.
\begin{theorem}\label{2.20}
    Let $X$ be a topological space. Then $\varnothing$ and $X$ are closed, arbitrary intersections of closed sets are closed, and finite unions of closed sets are closed.
\end{theorem}

{\it Proof.} Since $\varnothing$ and $X$ are open, their complements $X$ and $\varnothing$ are closed. Given a collection of closed sets $\{A_\alpha\}_{\alpha \in J}$, we have
$$X - \bigcap_{\alpha \in J} A_\alpha = \bigcup_{\alpha \in J} (X - A_\alpha).$$
$X - A_\alpha$ is open for each $\alpha$, hence the arbitrary union is open, meaning the arbitrary intersection is closed. Similarly, for closed sets $A_1, \cdots, A_n$,
$$X - \bigcup_{i=1}^nA_i = \bigcap_{i=1}^n(X- A_i),$$
and the right side is a finite intersection of open sets and thus is open, meaning the finite union is closed. $\Box$

\begin{theorem}\label{2.21}
    Let $Y$ be a subspace of $X$. Then a set $A$ is closed in $Y$ if and only if it equals the intersection of a closed set of $X$ with $Y.$
\end{theorem}
{\it Proof.} Suppose $A = C \cap Y$ where $C$ is closed in $X$. Then $X - C$ is open in $X$, so $(X - C) \cap Y = Y - A$ is open in $Y$. Thus $A$ is closed in $Y$.

Conversely if $A$ is closed in $Y$ then $Y-A$ is open in $Y$, so $Y - A = U \cap Y$ for some $U$ open in $X$. Then $X - U$ is closed in $X$ and $A = Y \cap (X - U)$, so $A$ is the intersection of a closed set in $X$ with $Y.$ $\Box$

\begin{theorem}\label{2.22}
    Let $Y$ be a subspace of $X$. If $A$ is closed in $Y$ and $Y$ is closed in $X$, then $A$ is closed in $X$.
\end{theorem}
{\it Proof.} Since $A$ is closed in $Y$, $Y-A$ is open in $Y$, and thus $Y - A = U \cap Y$ for some $U$ open in $X.$ Since $Y$ is closed in $X$, $X - Y$ is open in $X$. Then $(X-Y) \cup U = X - (U \cap Y) = X - A$ is open in $X$, so that $A$ is closed in $X$. $\Box$

\begin{definition}\label{2.23}
    Given a subset $A$ of a topological space $X$, we define the {\it interior} Int $A$ of $A$ as the union of all open sets contained in $A$, and the {\it closure} Cl $A$ or $\overline{A}$ of $A$ as the intersection of all closed sets containing $A$.
\end{definition}
We remark that Int $A \subseteq A \subseteq \overline{A}$, with equality between the first two arguments if $A$ is open and between the second two if $A$ is closed.

In general, is $X$ is a topological space with subspace $Y$, then the closure of $A$ in $X$ is different from the closure of $A$ in $Y$. However, we may express the closure in $Y$ in terms of the general closure, as the following theorem asserts.
\begin{theorem}\label{2.24}
    Let $Y$ be a subspace of $X$ and $A$ a subset of $Y$. Then the closure of $A$ in $Y$ is $\overline{A} \cap Y$, where $\overline{A}$ is the closure of $A$ in $X$.
\end{theorem}
{\it Proof.} The closure of $A$ in $Y$, $\overline{A}_Y$, is the intersection of all closed sets in $Y$ containing $A$. $\overline{A}$ is closed in $X$, and thus by \hyperref[2.21]{2.21} $\overline{A} \cap Y$ is closed in $Y$. Moreover $\overline{A} \cap Y$ contains $A$, so by definition, $\overline{A} \cap Y \subseteq \overline{A}_Y$.

Conversely, $\overline{A}_Y$ is closed in $Y$, so by \hyperref[2.21]{2.21} we may write $\overline{A}_Y = C \cap Y$ for some $C$ closed in $X$. In particular, $C$ must contain $A$, so $\overline{A} \subseteq C$, meaning $Y \cap \overline{A} \subseteq C \cap Y = \overline{A}_Y$. $\Box$

In the statement of the following theorem, $A$ {\it intersects} $B$ if their intersection $A\cap B$ is nonempty. Moreover, an open set containing $x$ is called a {\it neighbourhood} of $x$.
\begin{theorem}\label{2.25}
    Let $A$ be a subset of a topological space $X$. Then $x \in \overline{A}$ if and only if every neighbourhood of $x$ intersects $A$. Moreover, if $\mathcal{B}$ is a basis for $X$, then $x \in \overline{A}$ if and only if every $B \in \mathcal{B}$ containing $x$ intersects $A$.
\end{theorem}
{\it Proof.} We prove the contrapositive of both directions of the first part. Suppose $x \notin \overline{A}$. Then the open set $U = X - \overline{A}$ contains $x$ and does not intersect $A$. Conversely, if there exists an open set $U$ containing $x$ that does not intersect $A$, then $X - U$ is a closed set containing $A$, and thus $X - U$ contains $\overline{A}$, meaning $x \notin \overline{A}$.

To prove the second part, if every open set containing $x$ intersects $A$, then in particular this is true for basis elements. Conversely if every basis element containing $x$ intersects $A$, then so does every open set containing $x$, since there exists $B \in \mathcal{B}$ with $x \in B \subseteq U$, and thus $U \cap A \supseteq B \cap A$ is nonempty. $\Box$

The following examples seek to develop some intuition regarding closures.
\begin{itemize}
    \item Consider $A = (0, 1] \subseteq \reals.$ Every neighbourhood of $0$ intersects $A$, and clearly the same is true for any $x \in A$. However if $x \notin [0, 1]$, then there is a neighbourhood of $x$ disjoint from $A$, so $\overline{A} = [0, 1]$.
    \item Consider $B = \{\frac1n : n \in \ints^+\}$. Then $\overline{B} = \{0\} \cup B$, since every neighbourhood of $0$ contains some numbers in the form $\frac1n$.
    \item Consider $C = \{0\} \cup (1, 2)$. Then $\overline{C} = \{0\} \cup [1, 2]$.
    \item Consider $\rats$. By the density of the rationals in the reals, $\overline{\rats} = \reals$.
    \item Consider $\ints^+$. Unlike $\rats,$ if $x \notin \ints^+$, then there is a neighbourhood of $x$ containing no positive integers, so $\overline{\ints^+} = \ints^+$.
    \item Consider $\reals^+ = (0, \infty)$. Following the pattern of open intervals, we have $\overline{\reals^+} = \reals^+ \cup \{0\} = [0, \infty)$.
    \item Consider the subspace $Y = (0, 1]$ of $\reals$ and $D = (0, \frac12)$. The closure of $D$ in $\reals$ is $[0, \frac12]$; the closure of $D$ in $Y$ is $[0, \frac12] \cap Y = (0, \frac12]$.
\end{itemize}

Another way of describing the closure of a set involves the concept of a limit point.
\begin{definition}\label{2.26}
    If $A$ is a subset of a topological space $X$, a point $x$ in $X$ is a {\it limit point} of $A$ if every neighbourhood of $x$ intersects $A$ in some point other than $x$. Equivalently, $x$ is a limit point of $A$ if it belongs to the closure of $A - \{x\}$.
\end{definition}

In parallel with the examples of closures we saw, we find the limit points of various subsets of $\reals$.
\begin{itemize}
    \item Consider $A = (0, 1] \subseteq \reals.$ $0$ is the unique limit point not in $A$, while every point in $A$ is a limit point.
    \item Consider $B = \{\frac1n : n \in \ints^+\}$. $0$ is the unique limit point of $B$. Every other point $x \in \reals$ has a neighbourhood that either does not intersect $B$ or intersects $B$ only at $x$ itself.
    \item Consider $C = \{0\} \cup (1, 2)$. Then $[1, 2]$ is the set of limit points of $C$.
    \item Consider $\rats$. By the density of the rationals in the reals, the set of limit points of $\rats$ is $\reals$.
    \item Consider $\ints^+$. The set of limit points of $\ints^+$ is empty.
    \item Consider $\reals^+ = (0, \infty)$. The set of limit points of $\reals^+$ is $\{0\} \cup \reals^+$.
\end{itemize}
In comparison with the closures found for these subsets, one may have already guessed the result of the next theorem.
\begin{theorem}\label{2.27}
    Suppose $A$ is a subset of a topological space $X$, and $A'$ is the set of all limit points of $A$. Then $\overline{A} = A \cup A'$.
\end{theorem}
{\it Proof.} If $x \in A'$, then every neighbourhood of $x$ intersects $A$, so by \hyperref[2.25]{2.25} $x \in \overline{A}$. This means $A' \subseteq \overline{A}$, and since $A \subseteq \overline{A}$ by definition, we have $A \cup A' \subseteq \overline{A}$.

Conversely, suppose $x \in \overline{A}$. Moreover suppose $x \notin A$. Then by \hyperref[2.25]{2.25}, every neighbourhood $U$ of $x$ intersects $A$, but if $x \notin A$, $U$ must intersect $A$ at a point different from $x$, so that $x \in A' \subseteq A \cup A'$. $\Box$

We get a simple corollary.
\begin{corollary}\label{2.28}
    A subset of a topological space is closed if and only if it contains all its limit points.
\end{corollary}
{\it Proof.} $A$ is closed if and only if $A = \overline{A} = A \cup A'$, which is true if and only if $A' \subseteq A$. $\Box$

Open and closed sets and limit points in $\reals$ are perhaps overly ideal. For example, every one-point set is closed, but this is not generally true, such as in $\{a, b, c\}$. Furthermore, we say a sequence $x_1, x_2, \cdots$ of points in a topological space $X$ {\it converges} to a point $x \in X$ if for every neighbourhood $U$ of $x$, there exists $N \in \ints^+$ such that $x_n \in U$ for all $n > N$. In $\reals$ and $\reals^2$, a sequence converges to a unique point $x$, but this is not generally true. However, we would like to partially restrict our study of topological spaces to those that satisfy these geometric intuitions, bringing us to define a special class of topological spaces.
\begin{definition}\label{2.29}
    A topological space $X$ is {\it Hausdorff} if for each pair $x_1, x_2$ of distinct points in $X$, there exist neighbourhoods $U_1, U_2$ of $x_1, x_2$, respectively, that are disjoint.
\end{definition}
We may immediately prove that a Hausdorff space satisfies our first nice property.
\begin{theorem}\label{2.30}
    Every finite point set in a Hausdorff space $X$ is closed.
\end{theorem}
{\it Proof.} It suffices to show that every one-point set $\{x_0\}$ is closed, since finite unions of closed sets are closed. Suppose $x \in X$ is distinct from $x_0$, so that there exist disjoint neighbourhoods $U, U_0$ of $x, x_0$ , respectively. Since $U$ does not intersect $\{x_0\}$, $x$ is not in the closure of $\{x_0\}$, meaning the closure of $\{x_0\}$ is $\{x_0\}$ itself, and thus it is closed. $\Box$

Despite this simple proof, the Hausdorff condition is in fact stronger than the result that finite point sets be closed, called the {\it $T_1$ axiom}. For example, $\reals$ in the finite complement topology is not Hausdorff, but finite points are closed.
\begin{theorem}\label{2.31}
    Suppose $X$ is a space satisfying the $T_1$ axiom and $A$ a subset of $X$. Then $x$ is a limit point of $A$ if and only if every neighbourhood of $x$ contains infinitely many points of $A$.
\end{theorem}
{\it Proof.} If every neighbourhood of $x$ intersects $A$ in infinitely many points, it must intersect $A$ at some point other than $x$, so $x$ is a limit point of $A$.

Conversely, we proceed by contradiction. Suppose $x$ is a limit point of $A$, and some neighbourhood $U$ of $x$ intersects $A$ at finitely many points. Then $U$ intersects $A - \{x\}$ at finitely many points $x_1, \cdots, x_n$. By the $T_1$ axiom, $\{x_1, \cdots, x_n\}$ is closed, so its complement $X - \{x_1, \cdots, x_n\}$ is open. Then $U \cap (X - \{x_1, \cdots, x_n\})$ is a neighbourhood of $x$ that does not intersect $A - \{x\}$, contradicting the assumption that $x$ is a limit point of $A$. $\Box$

For most of our purposes, the $T_1$ axiom will not be strong enough. The next two theorem show the full power of the Hausdorff axiom.
\begin{theorem}\label{2.32}
    Suppose $X$ is a Hausdorff space. Then a sequence of points in $X$ converges to at most one point of $X$.
\end{theorem}
{\it Proof.} Suppose $\{x_n\}$ is a sequence of points in $X$ that converges to $x,$ and $y \neq x$. By the Hausdorff axiom, let $U, V$ be disjoint neighbourhoods of $x, y$, respectively. Since $U$ contains $x_n$ for all $n > N$ for some $N$, the same cannot be true for $V$, meaning $x_n$ does not converge to $y$. $\Box$

\begin{theorem}\label{2.33}
    Every simply ordered set is a Hausdorff space in the order topology. The product of two Hausdorff space is Hausdorff. A subspace of a Hausdorff space is Hausdorff.
\end{theorem}
{\it Proof.} Suppose $X$ is a simply ordered set in the order topology, and $x_1, x_2 \in X$; without loss of generality, assume $x_1 < x_2$. First let us assume there exists an element $x_0$ such that $x_1 < x_0 < x_2$. Then if $X$ has a minimal element $a_0$ and a maximal element $b_0$, then $[a_0, x_0), (x_0, b_0]$ are disjoint neighbourhoods of $x_1, x_2$, respectively. Otherwise $X$ either has no maximal element or no minimal element, meaning there exists an element $a$ less than $x_1$ or an element $b$ greater than $x_2$ so we may replace the above neighbourhoods with $(a, x_0)$ or $(x_0, b)$.

If no such $x_0$ exists, then $[a_0, x_2), (x_1, b_0]$, with appropriate modifications if $X$ has no maximal or minimal element, are disjoint neighbourhoods of $x_1, x_2$, respectively. Thus $X$ is Hausdorff.

Suppose $X, Y$ are Hausdorff spaces, and $X \times Y$ a topological space in the product topology. Suppose $x_1 \times y_1, x_2 \times y_2$ are distinct points of $X \times Y$. Let $U_1, U_2$ be disjoint neighbourhoods of $x_1, x_2$. Then $U_1 \times Y, U_2 \times Y$ are open in $X \times Y$, contain $x_1 \times y_1, x_2 \times y_2$ respectively, and $U_1 \times Y \cap U_2 \times Y = \varnothing$. Thus $X \times Y$ is Hausdorff.

Suppose $X$ is a Hausdorff space and $Y$ a subspace of $X$. Suppose $y_1, y_2$ are distinct points in $Y$. Viewed as distinct points in $X$, there exist disjoints neighbourhoods $U_1, U_2$ of $x_1, x_2$ in $X$. Then $U_1 \cap Y$ is open in $Y$, contains $y_1$, and is disjoint from $U_2 \cap Y$, which is open in $Y$ and contains $y_2$. Thus $Y$ is Hausdorff. $\Box$

\subsection{Continuous Functions}
The following definitions generalize the notion of continuous functions on the real line to functions on any topological space.

Given a function $f:X \rightarrow Y$ and a subset $V$ of $Y$, we define $f^{-1}(V) = \{x \in X: f(x) \in V\}$.
\begin{definition}\label{2.34}
    Given topological spaces $X, Y$, a function $f: X\rightarrow Y$ is {\it continuous} if for each open set $V$ in $Y$, $f^{-1}(V)$ is open in $X$.
\end{definition}
In practice, to show continuity of $f$ it suffices to show that the inverse image of every basis element of $Y$ is open. Given an arbitrary open set $V$ we may write $V = \bigcup B_\alpha$ for basis elements $B_\alpha$, and since
$$f^{-1} (V) = \bigcup f^{-1}(B_\alpha),$$
$f^{-1}(V)$ must be open if each $f^{-1}(B_\alpha)$ is open.

Furthermore, to show this, it suffices to show that the inverse image of each subbasis element of $Y$ is open, since any basis element $B$ is a finite intersection of subbasis elements $\{S_i\}$, and then
$$f^{-1}(B) = f^{-1}(S_1) \cap \cdots \cap f^{-1}(S_n)$$
shows that if every $f^{-1}(S_i)$ is open, then $f^{-1}(B)$ is open.

To show that the above definition is in fact equivalent to the $\varepsilon$-$\delta$ definition if $X, Y = \reals$, suppose $f: \reals \rightarrow \reals$ is continuous according to \hyperref[2.34]{2.34}. Then given any $x_0 \in \reals$ and any $\varepsilon > 0$, $V = (f(x_0)-\varepsilon, f(x_0)+\varepsilon)$ is open in $\reals$. Thus $f^{-1}(V)$ is open in $\reals$. Since $f^{-1}(V)$ contains $x_0$, it must contain some open interval $(a, b)$ around $x_0$. Taking $\delta = \min\{x_0-a, b-x_0\}$, we have $|f(x) - f(x_0)| < \varepsilon$ whenever $|x-x_0| < \delta$, since $x \in (a, b) \subseteq f^{-1}(V)$ must satisfy $f(x) \in V$.

Conversely, suppose $f: \reals \rightarrow \reals$ is continuous according to the $\varepsilon$-$\delta$ definition. Given any basis element $(a, b)$ and any $x_0 \in f^{-1}(a, b)$, let $\varepsilon = \min\{f(x_0)-a, b-f(x_0)\}$. Then there exists $\delta > 0$ such that $0< |x-x_0| < \delta$ implies $|f(x) - f(x_0)| < \varepsilon$ or $f(x) \in (a, b)$, so that $(x_0 - \delta, x_0 + \delta) \subseteq f^{-1}(a, b)$. Thus we have
$$f^{-1}(a, b) = \bigcup_{x \in f^{-1}(a, b)} (x - \delta_x, x + \delta_x),$$
since for any $x \in f^{-1}(a, b)$, 
$$x \in (x - \delta_x, x + \delta_x) \subseteq \bigcup_{x \in f^{-1}(a, b)} (x - \delta_x, x + \delta_x),$$
and for any $x \in \bigcup_{x \in f^{-1}(a, b)} (x - \delta_x, x + \delta_x)$, 
$$x \in (x - \delta_x, x + \delta_x) \subseteq f^{-1}(a, b).$$
The following example illustrates that the continuity of a function depends on the topology on the domain and codomain.

Consider the identity function $f: \reals \rightarrow \reals_\ell$. The inverse image of $[a, b) \subseteq \reals_\ell$ is clearly itself, but this is not open in $\reals$, so $f$ is not continuous. However, the identity function $g: \reals_\ell \rightarrow \reals$ is continuous because the inverse image of $(a,b)$ is itself which is open in $\reals_\ell$.

We now investigate various criteria for continuity.
\begin{theorem}\label{2.35}
    Let $f:X \times Y$, where $X$ and $Y$ are topological spaces. Then the following are equivalent:
    \begin{enumerate}
        \item[(1)] $f$ is continuous.
        \item[(2)] For every $x \in X$ and every neighbourhood $V$ of $f(x)$, there exists a neighbourhood $U$ of $x$ such that $f(U) \subseteq V$.
        \item[(3)] For every subset $A$ of $X$, $f(\overline{A}) \subseteq \overline{f(A)}$.
        \item[(4)] For every closed set $B$ of $Y$, $f^{-1}(B)$ is closed in $X$.
    \end{enumerate}
\end{theorem}
{\it Proof.} $(1) \implies (2)$. Suppose $f$ is continuous, $x \in X$, and $V$ is a neighbourhood of $f(x)$. By continuity, $U = f^{-1}(V)$ is open in $X$ and $x \in U$. Moreover, 
$$f(U) = f(f^{-1}(V)) = f(\{x \in X: f(x) \in V\}) \subseteq V.$$

$(2) \implies (3)$. Suppose $A \subseteq X$. For each $x \in \overline{A}$ and each neighbourhood $V$ of $f(x)$, there is a neighbourhood $U$ of $x$ such that $f(U) \subseteq V$. In particular, $U$ must intersect $A$ at some point $y$, so that
\begin{align*}
    f(y) &\in f(A \cap U), \\
    &= f(A) \cap f(U), \\
    &\subseteq f(A) \cap V,
\end{align*}
and thus $f(x) \in \overline{f(A)}$.

$(3) \implies (4)$. Suppose $B$ is closed in $Y$ and $A = f^{-1}(B)$. Since $f(A) = f(f^{-1}(B)) \subseteq B$, we have $\overline{f(A)} \subseteq B$. Thus if $x \in \overline{A}$, then
$$f(x) \in f(\overline{A}) \subseteq \overline{f(A)} \subseteq B,$$
showing that $x \in f^{-1}(B) = A$. Thus $\overline{A} \subseteq A$, so $A$ is closed.

$(4) \implies (1)$. If $V$ is open in $Y$, then $Y-V$ is closed in $Y$, meaning $f^{-1}(Y-V)$ is closed in $X$. Since
$$f^{-1}(Y-V) = \{x \in X: f(x) \in Y-V\} = X - \{x \in X:f(x) \in V\} = X - f^{-1}(V),$$
$f^{-1}(V)$ is open in $X$, showing that $f$ is continuous. $\Box$

The following definition generalizes is equivalent to isomorphisms between algebraic objects.
\begin{definition}\label{2.36}
    Suppose $X, Y$ are topological spaces and $f: X \rightarrow Y$ is a bijection with inverse $f^{-1}$. $f$ is a {\it homeomorphism} if both $f$ and $f^{-1}$ are continuous.
\end{definition}
Equivalently, $U \in X$ is open if and only if $f(U) \in Y$ is open. Thus we see that a homeomorphism is a bijection that preserves the topological structures, just like how isomorphisms preserve algebraic structures.

\begin{definition}\label{2.37}
    If $f: X \rightarrow Y$ is a continuous injection between topological spaces and the restriction of codomain $f': X \rightarrow f(X)$ is a homeomorphism, then $f$ is an {\it imbedding} of $X$ in $Y.$
\end{definition}
$f: \reals \rightarrow \reals$ defined by $f(x) = 3x+1$ is a homeomorphism; $f$ is bijective with inverse $f^{-1}(y) = \frac13(y-1)$, and both $f, f^{-1}$ are continuous.

$f:(-1, 1) \rightarrow \reals$ defined by $f(x) = \frac{x}{1-x^2}$ is a homeomorphism; its inverse is $f^{-1}(y) = \frac{2y}{1+(1+4y^2)^\frac12}$. To prove that $f$ is a homeomorphism, we may observe that $f$ is order-preserving and bijective, so $f$ maps every basis element of $(-1, 1)$ onto a basis element of $\reals$, and vice versa. Alternatively, we may show $\varepsilon$-$\delta$ continuity of $f, f^{-1}$.

As an example of a continuous bijection that is not a homeomorphism, consider the identity map $f: \reals_\ell \rightarrow \reals$. We previously saw that $f$ is continuous, but its inverse, the identity map $f^{-1}: \reals \rightarrow \reals_\ell$, is not.

As another example, let $S^1 = \{x \times y \in \reals^2 : x^2 + y^2 = 1\}$ denote the unit circle in the subspace topology and let $f: [0, 1) \rightarrow S^1$ be the defined by $f(t) = (\cos 2\pi t, \sin 2\pi t)$. $f$ is a continuous bijection by familiar properties of $\cos$ and $\sin$, but $f^{-1}$ is not continuous. Consider the image of $U = [0, \frac14) \subseteq [0, 1)$ under $f$. In particular, there is no neighbourhood $V$ of $f(0)$ in $\reals^2$ such that $V \cap S^1 \subseteq f(U)$, meaning $f(U)$ is not open in $S^1$.

This example also gives an example of a continuous injection that is not an imbedding: $g: [0, 1) \rightarrow \reals^2$ obtained by expanding the codomain of the previous function $f$.

Some familiar results about continuous functions in analysis are generalized below.
\begin{theorem}\label{2.38}
    Let $X, Y, Z$ be topological spaces. Then
    \begin{enumerate}
        \item[(a)] If $f: X \rightarrow Y$ maps all of $X$ into a single point $y_0 \in Y$, then $f$ is continuous.
        \item[(b)] If $A$ is a subspace of $X$, the inclusion function $j: A \rightarrow X$ is continuous.
        \item[(c)] If $f: X \rightarrow Y$ and $g: Y \rightarrow Z$ are continuous, then $g \circ f: X \rightarrow Z$ is continuous.
        \item[(d)] If $f: X \rightarrow Y$ is continuous and $A$ is a subspace of $X$, then $\restr{f}{A}: A \rightarrow Y$ is continuous.
        \item[(e)] If $f: X \rightarrow Y$ is continuous and $Z$ is a subspace of $Y$ containing $f(X)$ or $Z$ is a space having $Y$ as a subspace, then $f': X \rightarrow Z$ obtained by changing the codomain of $f$ is continuous.
        \item[(f)] $f: X \rightarrow Y$ is continuous if $X$ can be written as the union of open sets $U_\alpha$ such that $\restr{f}{U_\alpha}$ is continuous for each $\alpha$.
    \end{enumerate}
\end{theorem}
{\it Proof.} (a) Suppose $f(x) = y_0$ for every $x \in X$, and $V$ is open in $Y$. Then $f^{-1}(V) =X$ if $y_0 \in V$ and $\varnothing$ if $y_0 \notin V$. Thus $f^{-1}(V)$ is open.

(b) Suppose $U$ is open in $X$. Then $j^{-1}(U) = U \cap A$ is open in the subspace topology on $A$ by definition.

(c) Suppose $U$ is open in $Z$. Then $g^{-1}(U)$ is open in $Y$, so that $f^{-1}(g^{-1}(U)) = (g\circ f)^{-1}(U)$ is open in $X$.

(d) $\restr{f}{A}$ may be written as $j \circ f$, where $j: A \rightarrow X$ is the inclusion map. Thus by (c), $\restr{f}{A}$ is continuous.

(e) Suppose $f: X \rightarrow Y$ is continuous and $f(X) \subseteq Z \subseteq Y$; consider $g: X \rightarrow Z$ obtained from $f$. Let $W$ be open in $Z$. Then $W = Z \cap V$ for some $V$ open in $Y$. Since $f(X) \subseteq Z$, 
$$g^{-1}(W) = \{x \in X: g(x) \in W\} = \{x \in X: f(x) \in Z \cap V\} = \{x \in X: f(x) \in V\} = f^{-1}(V)$$
is open. Thus $g$ is continuous.

Alternatively if $Y \subseteq Z$, then $h: X \rightarrow Z$ obtained from $f$ may be written as $f \circ j$, where $j: Y \rightarrow Z$ is the inclusion map. Thus by (c), $h$ is continuous.

(f) Suppose $X = \bigcup_\alpha U_\alpha$, and $\restr{f}{U_\alpha}$ is continuous for each $\alpha$. Consider an open set $V$ in $Y$. Then
$$f^{-1}(V) \cap U_\alpha = \{x \in U_\alpha: f(x) \in V\} = (\restr{f}{U_\alpha})^{-1}(V)$$
is open in $U_\alpha$ by continuity of $\restr{f}{U_\alpha}$, and thus open in $X$. Therefore
$$f^{-1}(V) = \bigcup_\alpha (f^{-1}(V) \cap U_\alpha)$$
is open in $X$. $\Box$

The following theorem is known as the pasting lemma, and will be useful when we eventually explore algebraic topology.
\begin{theorem}\label{2.39}
    Suppose $X = A \cup B$, where $A, B$ are closed in $X$. Let $f: A \rightarrow Y, g: B \rightarrow Y$ be continuous. If $f(x) = g(x)$ for every $x \in A \cap B$, and $h: X \rightarrow Y$ is defined by
    $$h(x) = \begin{cases}
        f(x) &\text{if } x \in A, \\
        g(x) &\text{if } x\in B,
    \end{cases}$$
    Then $h$ is continuous.
\end{theorem}
{\it Proof.} Let $C$ be closed in $Y$. We see that
\begin{align*}
    h^{-1}(C) &= \{x \in X: h(x) \in C\}, \\
    &= \{x \in A \cup B: h(x) \in C\}, \\
    &= \{x \in A: f(x) \in C\} \cup \{x \in B: g(x) \in C\}, \\
    &= f^{-1}(C) \cup g^{-1}(C),
\end{align*}
and since $f^{-1}(C), g^{-1}(C)$ are closed in $A$, and thus closed in $X$, $h^{-1}(C)$ is closed in $X$. $\Box$

A similar statement holds if $A, B$ are open in $X$.

Let $h: \reals \rightarrow \reals$ be defined by
$$h(x) = \begin{cases}
    x &\text{for } x \leq 0, \\
    \frac{x}{2} &\text{for } x \geq 0.
\end{cases}$$
$\reals = A \cup B$ where $A = (-\infty, 0], B = [0, \infty)$ are closed in $\reals$. $x$ is continuous on $A$, $\frac{x}{2}$ is continuous on $B$, and $x = \frac{x}{2}$ on $A \cap B = \{0\}$, thus by \hyperref[2.39]{2.39}, $h$ is continuous on $\reals$.

The criterion that $f(x) = g(x)$ for every $x \in A \cap B$ is necessary for $h$ to define a function. For example, if
$$k(x) = \begin{cases}
    x-2 &\text{for } x \leq 0, \\
    x+2 &\text{for } x \geq 0,
\end{cases}$$
then $k(0) = -2$ and $k(0) = 2$, so $k$ is not even a function.

Moreover, the criterion that $A, B$ are closed is equally necessary. 
$$l(x) = \begin{cases}
    x-2 &\text{for } x < 0, \\
    x+2 &\text{for } x \geq 0
\end{cases}$$
defines a valid function $l: \reals \rightarrow \reals$, but $l$ is not continuous as $f^{-1}(1, 3) = [0, 1)$, which is not open.

The next theorem provides a useful criterion for continuity of maps into products.
\begin{theorem}\label{2.40}
    Let $f: A \rightarrow X \times Y$ be given by $f(a) = (f_1(a), f_2(a))$. Then $f$ is continuous if and only if the coordinate functions $f_1: A \rightarrow X, f_2: A \rightarrow Y$ are both continuous.
\end{theorem}
{\it Proof.} $(\implies)$ We first show that the projections $\pi_1, \pi_2$ are continuous. Indeed, $\pi_1^{-1}(U) = U \times Y$ and $\pi_2^{-1} = X \times V$ are open in $X \times Y$ provided that $U, V$ are open. Given that
$$f_1 = \pi_1 \circ f \text{ and } f_2 = \pi_2 \circ f,$$
continuity of $f$ implies continuity of $f_1, f_2$ by composition of continuous functions. 

$(\impliedby)$ Conversely, if $f_1, f_2$ are continuous, then for each basis element $U \times V$ of $X \times Y$, consider $f^{-1}(U \times V)$. By definition, $a \in f^{-1}(U \times V)$ means $f(a) \in U \times V$, or equivalently, $f_1(a) \in U$ and $f_2(a) \in V$. Thus
$$f^{-1}(U \times V) = f_1^{-1}(U) \cap f_2^{-1}(V)$$
is an intersection of open sets, hence is open. $\Box$

A parameterized curve in $\reals^2$ is a continuous map $f: [a, b] \rightarrow \reals^2$. By \hyperref[2.40]{2.40}, $f(t) = (x_1(t), x_2(t))$ is continuous if and only if $x_1(t), x_2(t)$ are both continuous.

\subsection{The Product Topology, revisited}
We now generalize our previous survey of product topologies to arbitrary Cartesian products.

\begin{definition}\label{2.41}
    Given a countable collection $\{X_i\}$ of topological spaces, we define the {\it box topology} on the Cartesian products
    $$X_1 \times \cdots \times X_n \text{ or } X_1 \times X_2 \times \cdots$$
    by taking as basis all sets of the form
    $$U_1 \times \cdots \times U_n \text{ or } U_1 \times U_2 \times \cdots$$
    where $U_i$ is open in $X_i$.
\end{definition}

However, the following definition will generally be more useful.
\begin{definition}\label{2.42}
    Given a countable collection $\{X_i\}$ of topological spaces, we define the {\it product topology} on the Cartesian products
    $$X_1 \times \cdots \times X_n \text{ or } X_1 \times X_2 \times \cdots$$
    by taking as subbasis all sets of the form $\pi_i^{-1}(U_i)$, where $U_i$ is open in $X_i$.
\end{definition}
Let us compare the box topology with the product topology. The general basis element for the product topology is a finite intersection of subbasis elements 
$$\pi_i^{-1}(U_i) = X_1 \times \cdots \times U_i \times \cdots \times X_n \text{ or } X_1 \times \cdots \times U_i \times \cdots,$$
and thus is in the form
$$X_1 \times \cdots \times U_{i_1} \times \cdots \times U_{i_k} \times \cdots \times X_n \text{ or } X_1 \times \cdots \times U_{i_1} \times \cdots \times U_{i_k} \times \cdots$$
for some finite sequence $\{U_{i_j}\}_{j=1}^k$ of open sets in $X_{i_j}$. Since each $X_i$ is open in $X_i$, any basis element in the product topology is also a basis element in the box topology for countable Cartesian products. Conversely, for finite Cartesian products only, every basis element in the box topology is a basis element in the product topology.

We may furthermore generalize to arbitrary Cartesian products.
\begin{definition}\label{2.43}
    Given an index set $J$ and a set $X$, we define a {\it $J$-tuple} of elements of $X$ to be a function $\mathbf{x}: J \rightarrow X$. For $\alpha \in J$, $x_\alpha = \mathbf{x}(\alpha)$ is the $\alpha$th coordinate of $\mathbf{x}$, and we note $\mathbf{x}$ by $(x_\alpha)_{\alpha \in J}$. Finally, $X^J$ is the set of all $J$-tuples of elements of $X$.
\end{definition}

Now that we have appropriate notation for $J$-tuples, we may define Cartesian products of indexed families of sets.
\begin{definition}\label{2.44}
    Let $\{A_\alpha\}_{\alpha \in J}$ be an indexed family of sets. The Cartesian product $\prod_{\alpha \in J}A_\alpha$ of $\{A_\alpha\}_{\alpha \in J}$ is the set of all $J$-tuples $(x_\alpha)_{\alpha \in J}$ of elements of $\bigcup_{\alpha \in J} A_\alpha$ such that $x_\alpha \in A_\alpha$ for each $\alpha \in J$.
\end{definition}
As an example, consider the indexed family of sets $\{(-\infty, a]\}_{a \in \reals}$. Since $\bigcup_{a \in \reals} (-\infty, a] = \reals$, $\prod_{a\in \reals} (-\infty, a]$ is the set of all functions $f: \reals \rightarrow \reals$ such that $f(a) \in (-\infty, a]$ for each $a \in \reals$, or all functions entirely lying below or on the line $y = x$.

We remark that if each $A_\alpha$ is the same set $A$, then $\prod_{\alpha \in J} A_\alpha = A^J$. Now, we are sufficiently prepared to define topologies on these Cartesian products.
\begin{definition}\label{2.45}
    Given an indexed family of topological spaces $\{X_\alpha\}_{\alpha \in J}$, we define the {\it box topology} on $\prod_{\alpha \in J}X_\alpha$ by taking as basis the collection of all sets of the form $\prod_{\alpha \in J}U_\alpha$, where $U_\alpha$ is open in $X_\alpha$.
\end{definition}
This basis satisfies the first condition since $\prod_{\alpha \in J}X_\alpha$ itself is a basis element, and the second is satisfied since the intersection of two basis elements
$$\left( \prod_{\alpha \in J} U_\alpha \right) \cap \left( \prod_{\alpha \in J}V_\alpha\right) = \prod_{\alpha \in J} (U_\alpha \cap V_\alpha)$$
is again a basis element.

\begin{definition}\label{2.46}
    Let the projection mapping $\pi_\beta: \prod_{\alpha \in J}X_\alpha \rightarrow X_\beta$ associated with the index $\beta$ be defined by $(x_\alpha)_{\alpha \in J} \mapsto x_\beta$, and let $\mathcal{S}_\beta$ be defined by
    $$\mathcal{S}_\beta = \{\pi_\beta^{-1}(U_\beta): U_\beta \text{ open in } X_\beta\}.$$

    Given an indexed family of topological spaces $\{X_\alpha\}_{\alpha \in J}$, we define the {\it product topology} on $\prod_{\alpha \in J}X_\alpha$ by taking as subbasis $\mathcal{S} = \bigcup_{\beta \in J} \mathcal{S}_\beta$.
\end{definition}
$\mathcal{S}$ is clearly a subbasis since $\pi_\beta^{-1}(X_\beta) = \prod_{\alpha \in J}X_\alpha \in \mathcal{S}_\beta \subseteq \mathcal{S}$.

The basis $\basis$ generated by $\mathcal{S}$ is the collection of finite intersections of elements in $\mathcal{S}$. The intersection of elements belonging to the same $S_\beta$ is
$$\pi_\beta^{-1}(U_\beta) \cap \pi_\beta^{-1}(V_\beta) = \pi_\beta^{-1}(U_\beta \cap V_\beta),$$
which is again in $S_\beta$. However, the intersection of elements belonging to different $S_\beta$ generates new elements not in $\mathcal{S}$. The general basis element takes the form
$$B = \pi_{\beta_1}^{-1}(U_{\beta_1}) \cap \cdots \cap \pi_{\beta_n}^{-1}(U_{\beta_n}),$$
where $\beta_1, \cdots, \beta_n \in J$ is a finite set of distinct indices and $\{U_{\beta_i}\}_{\beta_i \in J}$ is a sequence of sets open in $X_{\beta_i}$. Equivalently, we may write $B = \prod_{\alpha \in J}U_\alpha$, where $U_\alpha$ is open in $X_\alpha$ for each $\alpha$ and $U_\alpha = X_\alpha$ for all but finitely many values of $\alpha$. From this, it is clear that for finite products, the box and product topologies are the same, and otherwise the box topology is strictly finer.

From this point onwards, we will assume any Cartesian product of topological spaces is given the product topology. The next few theorems will generalize our previous survey of products $X \times Y$.

\begin{theorem}\label{2.47}
    Suppose $\{X_\alpha\}$ is an indexed family of topological spaces with corresponding bases $\mathcal{B}_\alpha$. The collection of all sets of the form $\prod_{\alpha \in J} B_\alpha$, where $B_\alpha \in \mathcal{B}_\alpha$ for each $\alpha$, is a basis for the box topology on $\prod_{\alpha \in J}X_\alpha$, and the collection of all sets in this form where $B_\alpha \in \mathcal{B}_\alpha$ for finitely many $\alpha$ and $B_\alpha = X_\alpha$ for the remaining $\alpha$ is a basis for the product topology.
\end{theorem}
{\it Proof.} Given $x \in \prod_{\alpha \in J} X_\alpha$ and a neighbourhood $U$ of $x$ in the box topology, there is a basis element $\prod_{\alpha \in J} U_\alpha$ containing $x$ that is contained in $U$. Each $U_\alpha$ is open by definition, so there exists $B_\alpha \in \mathcal{B}_\alpha$ such that $x_\alpha \in B_\alpha \subseteq U_\alpha$. Then $x \in \prod_{\alpha \in J} B_\alpha \subseteq \prod_{\alpha \in J} U_\alpha$, and thus the collection of all sets of the form $\prod_{\alpha \in J} B_\alpha$ is a basis for a topology finer than the box topology. Conversely, this collection is contained in the basis for the box topology, so the topology it generates is precisely the box topology.

Similarly, for each $x$ and each neighbourhood of $x$ in the product topology, there exists $B = \prod_{\alpha \in J} U_\alpha$ such that $x \in B \subseteq U$. For the finitely many values of $\alpha$ such that $U_\alpha = X_\alpha,$ let $B_\alpha \in \mathcal{B}_\alpha$ be such that $x_\alpha \in B_\alpha \subseteq U_\alpha$. For the remaining $\alpha$, let $B_\alpha = X_\alpha$. Then $x \in \prod_{\alpha \in J} B_\alpha \subseteq B$, so that the collection of all sets in the form $\prod_{\alpha \in J} B_\alpha$ generates a topology finer than the product topology. Since this basis is contained in the basis for the product topology, they are in fact equal. $\Box$

A simple example of a finite Cartesian product is the $n$-dimensional Euclidean space $\reals^n$. A basis for $\reals$ consists of all open intervals in $\reals$, so a basis for $\reals^n$ consists of all products of the form 
$$(a_1, b_1) \times \cdots \times (a_n, b_n).$$
Since $\reals^n$ is finite, the box and product topologies agree.

\begin{theorem}\label{2.48}
    Let $A_\alpha$ be a subspace of $X_\alpha$ for each $\alpha \in J$. Then $\prod_{\alpha \in J} A_\alpha$ is a subspace of $\prod_{\alpha \in J} X_\alpha$ if both products are given the box topology or the product topology.
\end{theorem}
{\it Proof.} Suppose $\prod X_\alpha$ is given the box topology and $\prod U_\alpha$ is open in $\prod X_\alpha$. Equivalently, $U_\alpha$ is open in $X_\alpha$, and thus $U_\alpha \cap A_\alpha$ is open in $A_\alpha$. Since $\prod U_\alpha \cap \prod A_\alpha = \prod (U_\alpha \cap A_\alpha)$, the subspace topology (left) and the box topology (right) on $\prod A_\alpha$ are the same. 

If $\prod X_\alpha$ is given the product topology, then all but finitely many $U_\alpha$ are equal to $X_\alpha$ so that $U_\alpha \cap A_\alpha = A_\alpha$ for these values $\alpha$. Thus $\prod (U_\alpha \cap A_\alpha)$ is the product topology, and coincides with the subspace topology. $\Box$

\begin{theorem}\label{2.49}
    If each space $X_\alpha$ is Hausdorff, then $\prod X_\alpha$ is Hausdorff in the box and product topologies.
\end{theorem}
{\it Proof.} Given distinct points $(x_\alpha), (y_\alpha) \in \prod X_\alpha$, there must exist $\beta$ such that $x_\beta \neq y_\beta$. Since $X_\beta$ is Hausdorff, let $U_\beta, V_\beta$ be disjoint neighbourhoods of $x_\beta, y_\beta$. Then $\pi_\beta^{-1}(U_\beta), \pi_\beta^{-1}(V_\beta)$ are disjoint neighbourhoods of $(x_\alpha), (y_\alpha)$. $\Box$

\begin{theorem}\label{2.50}
    Let $\{X_\alpha\}$ be an indexed family of spaces and $A_\alpha \subseteq X_\alpha$ for each $\alpha$. If $\prod X_\alpha$ is given the box topology or the product topology, then $\prod \overline{A_\alpha} = \overline{\prod A_\alpha}$.
\end{theorem}
{\it Proof.} Let $(x_\alpha) \in \prod \overline{A_\alpha}$ and let $\prod U_\alpha$ be a basis element for either the box or product topology containing $(x_\alpha)$. For each $\alpha$, $x_\alpha \in \overline{A_\alpha}$, so there exists $y_\alpha \in U_\alpha \cap A_\alpha$. Then $(y_\alpha) \in \prod U_\alpha \cap \prod A_\alpha$, so that $x \in \overline{\prod A_\alpha}$.

Conversely, let $(x_\alpha) \in \overline{\prod A_\alpha}$. For any $\beta$, let $V_\beta$ be a neighbourhood of $x_\beta$ in $X_\beta$. Since $\pi_\beta^{-1}(V_\beta)$ is a neighbourhood of $(x_\alpha)$ in $\prod X_\alpha$, there exists $(y_\alpha) \in \pi_\beta^{-1}(V_\beta) \cap \prod A_\alpha$. Then $y_\beta \in V_\beta \cap A_\beta$, so that $x_\beta \in \overline{A_\beta}$ for each $\beta$. $\Box$

The following theorem justifies our preference for the product topology, since it is false in the box topology.
\begin{theorem}\label{2.51}
    Let $\prod X_\alpha$ be given the product topology and suppose $f: A \rightarrow \prod X_\alpha$ is defined by 
    $$f(a) = (f_\alpha(a))$$
    where $f_\alpha: A \rightarrow X_\alpha$ for each $\alpha$. Then $f$ is continuous if and only if each $f_\alpha$ is continuous.
\end{theorem}

{\it Proof.} We remark that $\pi_\beta$ is continuous since given $U_\beta$ open in $X_\beta$, $\pi_\beta^{-1}(U_\beta)$ is a subbasis element for the product topology on $\prod X_\alpha$. From this, if $f$ is continuous then $f_\beta = \pi_\beta \circ f$ is continuous for each $\beta$.

Conversely, if each $f_\alpha$ is continuous, then we show that the inverse image of an arbitrary subbasis element $\pi_\beta^{-1}(U_\beta)$ is open in $A$. Since
$$f^{-1}(\pi_\beta^{-1}(U_\beta)) = f_\beta^{-1}(U_\beta),$$
and $f_\beta$ is continuous, this set is open in $A$. $\Box$

As a counterexample in the box topology, consider 
$$\reals^\omega = \prod_{n \in \ints^+} X_n,$$
where $X_n = \reals$ for each $n$. Let $f: \reals \rightarrow \reals^\omega$ be defined by
$$f(t) = (t, t, \cdots),$$
so that $f_n(t) = t$ for each $n$. We know that each $f_n$ is continuous, hence $f$ is continuous in the product topology. However, in the box topology, consider the basis element
$$B = (-1, 1) \times (-\frac12 , \frac12) \times (-\frac13, \frac13) \times \cdots.$$
If $f^{-1}(B)$ were open in $\reals$, then it would contain some neighbourhood $(-\delta, \delta)$ of $0$. Then $f(-\delta, \delta) \subseteq B$. Taking $N > \frac{1}{\delta}$ and applying $\pi_N$ to both sides results in
$$f_N(-\delta, \delta) = (-\delta, \delta) \subseteq (-\frac{1}{N}, \frac{1}{N}),$$
a contradiction.

\subsection{The Metric Topology}
\begin{definition}\label{2.52}
    A {\it metric} on a set $X$ is a function $d: X \times X \rightarrow \reals$ having the following properties:
    \begin{enumerate}
        \item[(1)] (Positivity) $d(x, y) \geq 0$ for all $x, y \in X$, with equality if and only if $x = y$.
        \item[(2)] (Symmetry) $d(x, y) = d(y, x)$ for all $x, y \in X$.
        \item[(3)] (Triangle inequality) $d(x, y) + d(y, z) \geq d(x, z)$ for all $x, y, z \in X.$
    \end{enumerate}
    Given a metric $d$ on $X$, $d(x, y)$ is called the {\it distance} between $x$ and $y$ in the metric $d$.
\end{definition}
Given a number $\varepsilon > 0$, the set
$$B_d(x, \varepsilon) = \{y: d(x, y) < \varepsilon\}$$
is called the $\varepsilon$-{\it ball centered at } $x$. From this, we obtain a canonical way to construct a topology on a set with a metric.
\begin{definition}\label{2.53}
    Given a set $X$ with metric $d$, the {\it metric topology} on $X$ induced by $d$ is defined by taking as basis the collection of all $\varepsilon$-balls for $x \in X$ and $\varepsilon > 0$.
\end{definition}
The first basis condition is immediate from the fact that any $x \in X$ is contained in any ball centered at $x$ itself. To show the second condition, we will use the following lemma.
\begin{lemma}\label{2.54}
    Suppose $B_d(x, \varepsilon)$ is a ball and $y \in B_d(x, \varepsilon)$. Then there exists $\delta > 0$ such that $B_d(y, \delta) \subseteq B_d(x, \varepsilon)$.
\end{lemma}
{\it Proof.} Let $\delta = \varepsilon - d(x, y)$. If $z \in B(y, \delta)$, then $d(y, z) < \varepsilon - d(x, y)$, and by the triangle inequality,
$$d(x, z) \leq d(x, y) + d(y, z) < \varepsilon$$
so that $z \in B(x, \varepsilon)$. $\Box$

Now, to show the second condition, suppose $B_1, B_2$ are two basis elements such that $y \in B_1 \cap B_2$. Then there exist $\delta_1, \delta_2 > 0$ such that $B(y, \delta_1) \subseteq B_1, B(y, \delta_2) \subseteq B_2$. Taking $\delta = \min\{\delta_1, \delta_2\}$, we obtain $B(y, \delta) \subseteq B_1 \cap B_2$.

From this construction, we see that a set $U$ is open if and only if for each $y \in U$, there exists $\delta > 0$ such that $B_d(y, \delta) \subseteq U$. As a sanity check, this condition implies $U = \bigcup_{y \in U} B_d(y, \delta)$ is open. Conversely, if $U$ is open, then it contains a basis element $B_d(x, \varepsilon)$ containing $y$, which contains a basis element $B_d(y, \delta)$ centered at $y$ by \hyperref[2.54]{2.54}.

As an example of a metric on any set $X$ is
\begin{align*}
    d(x, y) = 1 &\text{ if } x\neq y, \\
    d(x, y) = 0 &\text{ if } x=y.
\end{align*}
In fact, $d$ induces the discrete topology since the basis element $B(x, 1)$ consists only of the point $x$.

The standard metric for $\reals$ is $d(x, y) = |x-y|$, which induces the order topology as every basis element $(a, b)$ may be written as $B\left(\frac{a+b}{2}, \frac{b-a}{2}\right)$, and every basis element $B(x, \varepsilon)$ may be written as $(x-\varepsilon, x + \varepsilon)$.

These examples show that some familiar topologies may be represented as by metric topologies. The following definition gives this class of topologies a name.
\begin{definition}\label{2.55}
    A topological space $X$ is {\it metrizable} if there exists a metric $d$ on $X$ which induces the topology of $X$. A {\it metric space} is a metrizable space along with metric $d$.
\end{definition}
An important quality of metric spaces in analysis is the following:
\begin{definition}\label{2.56}
    A subset $A$ of a metric space $X$ with metric $d$ is {\it bounded} if there exists $M$ such that $d(a_1, a_2) \leq M$ for all $a_1, a_2 \in A$. If $A$ is bounded and nonempty, the {\it diameter} of $A$ is defined as
    $$\text{diam}(A) = \sup \{d(a_1, a_2): a_1, a_2 \in A\}.$$
\end{definition}

Although boundedness is of little importance for topology's purposes, the following theorem gives a metric with respect to which every subset of $X$ is bounded.
\begin{theorem}\label{2.57}
    Given a metric space $X$ with metric $d$, we define the {\it standard bounded metric} $\overline{d}: X \times X \rightarrow \reals$ by $\overline{d}(x, y) = \min \{d(x, y), 1\}$. $\overline{d}$ is a metric that induces the same topology as $d$, while every subset of $X$ is bounded under $\overline{d}$.
\end{theorem}
{\it Proof.} $\overline{d}$ is clearly positive and symmetric since $d$ is a metric. If $d(x, y) \geq 1$ or $d(y, z) \geq 1$, then $\overline{d}(x, y) + \overline{d}(y, z) \geq 1$, thus the triangle inequality follows from the fact that $\overline{d}(x, z) \leq 1$. Otherwise $d(x, y) < 1$ and $d(y, z) < 1$, so
$$\overline{d}(x, y) + \overline{d}(y, z) = d(x, y) + d(y, z) \geq d(x, z) \geq \overline{d}(x, z).$$

Let $\topo, \topo'$ be the topologies induced by $d, \overline{d}$, respectively. $\topo'$ is generated by the collection of $\varepsilon$-balls with $\varepsilon < 1$, so clearly $\topo' \subseteq \topo$. To show that $\topo' \supseteq \topo$, given any $\varepsilon$-ball $B$ containing a point $x$, there exists a $\varepsilon$-ball with $\varepsilon < 1$ contained in $b$ that contains $x$. Thus $\topo = \topo'$.

Clearly every subset of $X$ in $\topo'$ is bounded by $1$, since $\overline{d}(x, y) \leq 1$ for all $x, y \in X$. $\Box$

By considering $\reals^n$ as an inner product space, we obtain two canonical metrics.
\begin{definition}\label{2.58}
    The {\it Euclidean metric} $d$ on $\reals^n$ is defined by
    $$d(\mathbf{x}, \mathbf{y}) = \|\mathbf{x} - \mathbf{y}\| = \sqrt{(x_1-y_1)^2 + \cdots + (x_n-y_n)^2},$$
    and the {\it square metric} $\rho$ is defined by
    $$\rho(\mathbf{x}, \mathbf{y}) = \max \{|x_1-y_1|, \cdots, |x_n-y_n|\}.$$
\end{definition}
It follows from (hopefully) familiar results of linear algebra that $d$ is positive, symmetric, and satisfies the triangle inequality. To see that $\rho$ is a metric, the first two properties are trivial, and the triangle inequality follows from the fact that
$$|x_i-z_i|\leq |x_i-y_i|+|y_i-z_i| \leq \rho(\mathbf{x}, \mathbf{y}) + \rho(\mathbf{y}, \mathbf{z})$$
for each $i$. We now wish to show that the Euclidean metric and the square metric induce the product topology on $\reals^n$.

The following lemma will be necessary:
\begin{lemma}\label{2.59}
    Suppose $d, d'$ are two metrics on $X$ inducing topologies $\topo, \topo'$. Then $\topo'$ is finer than $\topo$ if and only if for every $x \in X$ and every $\varepsilon > 0$, there exists $\delta > 0$ such that $B_{d'}(x, \delta) \subseteq B_d(x, \varepsilon).$
\end{lemma}
{\it Proof.} Suppose $\topo' \supseteq \topo$. Given a basis element $B_d(x, \varepsilon)$ of $\topo$, \hyperref[2.5]{2.5} implies that there exists a basis element $B'$ of $\topo'$ such that $x \in B' \subseteq B_d(x, \varepsilon)$. Then by \hyperref[2.54]{2.54}, there exists $\delta > 0$ such that
$$B_{d'}(x, \delta) \subseteq B' \subseteq B_d(x, \varepsilon).$$
Conversely, suppose the $\delta$-$\varepsilon$ condition holds; given $x \in X$ and a basis element $B \in \topo$ containing $x$, there exists a ball $B_d(x, \varepsilon)$ contained in $B$. By the $\delta$-$\varepsilon$ condition, there exists a basis element $B_{d'}(x, \delta) \in \topo'$ contained in $B_d(x, \varepsilon) \subseteq B$ and $B_{d'}(x, \delta)$ clearly contains $x$. Thus by \hyperref[2.5]{2.5}, $\topo' \supseteq \topo$. $\Box$

This lemma leads to an important theorem about the aforementioned metrics on $\reals^n$.

\begin{theorem}\label{2.60}
    The Euclidean metric $d$ and the square metric $\rho$ induce the product topology on $\reals^n$.
\end{theorem}
{\it Proof.} Let $\mathbf{x} = (x_1, \cdots, x_n), \mathbf{y} = (y_1, \cdots, y_n) \in \reals^n$. We remark that
$$\rho(\mathbf{x}, \mathbf{y}) \leq d(\mathbf{x}, \mathbf{y}) \leq \sqrt{n} \rho(\mathbf{x}, \mathbf{y}).$$
The first inequality implies $B_d(\mathbf{x}, \varepsilon) \subseteq B_\rho(\mathbf{x}, \varepsilon)$ for any $\mathbf{x} \in \reals^n$ and $\varepsilon > 0$ and the second implies $B_\rho(\mathbf{x}, \frac{\varepsilon}{\sqrt{n}}) \subseteq B_d(\mathbf{x}, \varepsilon)$. By \hyperref[2.59]{2.59}, they induce the same topology.

To show that the product topology coincides with the topology induced by $\rho$, consider $\mathbf{x} \in \reals^n$ and a basis element $B= (a_1, b_1) \times \cdots \times (a_n, b_n)$ for the product topology containing $\mathbf{x}$. For each $i$, there must exist $\varepsilon_i$ such that $(x_i-\varepsilon_i, x_i+\varepsilon_i) \subseteq (a_i, b_i)$. Let $\varepsilon = \min_i\{\varepsilon_i\}$. If $\mathbf{y} \in B_\rho(\mathbf{x}, \varepsilon)$, then
$$|y_i - x_i| \leq \max \{|x_1-y_1|, \cdots, |x_n-y_n|\} = \rho(\mathbf{x}, \mathbf{y}) < \varepsilon \leq \varepsilon_i,$$
and thus $y_i \in (x_i- \varepsilon_i, x_i + \varepsilon_i) \subseteq (a_i, b_i)$ for each $i$. Thus $\mathbf{x} \in B_\rho(\mathbf{x}, \varepsilon) \subseteq B$, showing that the $\rho$-topology is finer than the product topology.

Conversely, any basis element $B_\rho(\mathbf{x}, \varepsilon)$ for the $\rho$-topology may be written as
$$B_\rho(\mathbf{x}, \varepsilon) = (x_i - \varepsilon, x_i+\varepsilon) \times \cdots \times (x_n - \varepsilon, x_n + \varepsilon),$$
since $\mathbf{y} \in B_\rho(\mathbf{x}, \varepsilon)$ implies
$$|y_i - x_i| \leq \max\{|x_1 - y_1|, \cdots, |x_n - y_n| = \rho(\mathbf{x}, \mathbf{y}) < \varepsilon,$$
or $y_i \in (x_i - \varepsilon, x_i + \varepsilon)$ for each $i$. Thus as every basis element for the $\rho$-topology is a basis element for the product topology, the product topology must be finer. $\Box$

To define a metric on $\reals^\omega$, we will generalize the standard bounded metric $\overline{d}$.
\begin{definition}\label{2.61}
    Given an index set $J$ and points $\mathbf{x} = (x_\alpha)_{\alpha \in J}, \mathbf{y} = (y_\alpha)_{\alpha \in J} \in \reals^J$, we define the {\it uniform metric} $\overline{\rho}$ on $\reals^J$ by
    $$\overline{\rho}(\mathbf{x}, \mathbf{y}) = \sup_{\alpha \in J} \{\overline{d}(x_\alpha, y_\alpha)\},$$
    where $\overline{d}$ is the standard bounded metric on $\reals$. The uniform metric indues the uniform topology on $\reals^J$.
\end{definition}
As $0 \leq \overline{d}(x_\alpha, y_\alpha) \leq 1$ for each $\alpha$, $0 \leq \sup_{\alpha \in J} \{\overline{d}(x_\alpha, y_\alpha)\} \leq 1$, with equality at $0$ if and only if $\overline{d}(x_\alpha, y_\alpha) = 0$ for each $\alpha$, meaning $\mathbf{x} = \mathbf{y}$. Symmetry and the triangle inequality follow immediately from these properties of $\overline{d}$.

We may now compare the various topologies on $\reals^J$.
\begin{theorem}\label{2.62}
    The uniform topology on $\reals^J$ is finer than the product topology and coarser than the box topology, with strict containment in both instances if $J$ is infinite.
\end{theorem}
{\it Proof.} Let $\mathbf{x} = (x_\alpha)_{\alpha \in J} \in \reals^J$ and consider a basis element $\prod U_\alpha$ for the product topology containing $\mathbf{x}$. By definition, $U_\alpha = \reals$ for all but finitely many $\alpha$, which we denote $\alpha_1, \cdots, \alpha_n$. Since $U_{\alpha_i}$ is open in $\reals$ for $i = 1, \cdots, n,$ there exists a basis element $B$ in $\reals$ containing $x$ that is contained in $U_{\alpha_i}$, and by \hyperref[2.54]{2.54}, there exists $\varepsilon_i > 0$ such that $B_{\overline{d}}(x_{\alpha_i}, \varepsilon_i) \subseteq B \subseteq U_{\alpha_i}$. Let $\varepsilon = \min_{1 \leq i \leq n} \{\varepsilon_i\}$, so that $\mathbf{y} \in B_{\overline{\rho}}(\mathbf{x}, \varepsilon)$ implies
$$\overline{d}(x_{\alpha_i}, y_{\alpha_i}) \leq \overline{\rho}(\mathbf{x}, \mathbf{y}) < \varepsilon \leq \varepsilon_i$$
for each $\alpha_i$. Since $y_{\alpha_i} \in B_{\overline{d}}(x_{\alpha_i}, \varepsilon_i) \subseteq U_{\alpha_i}$ and surely $y_\alpha \in U_\alpha$ whenever $U_\alpha = \reals$, $\mathbf{y} \in \prod U_\alpha$. As $B_{\overline{\rho}}(\mathbf{x}, \varepsilon) \subseteq \prod U_\alpha$, the uniform topology is finer than the product topology.

Assuming $J$ is infinite, consider a basis element $B_{\overline{\rho}}(\mathbf{x}, \varepsilon)$ for the uniform topology, where $\varepsilon < 1$. Then there is no basis element $\prod U_\alpha$ for the product topology contained in $B_{\overline{\rho}}(\mathbf{x}, \varepsilon)$. In particular, define $\mathbf{y} \in \prod U_\alpha$ by $y_\alpha = x_\alpha$ for all $\alpha$ except one of the infinitely many $\beta$ for which $U_\beta = \reals$; let $|x_\beta - y_\beta| \geq 1$ so that $\overline{d}(x_\alpha, y_\alpha) = 1$. Then
$$\overline{\rho}(\mathbf{x}, \mathbf{y}) = 1,$$
meaning $\mathbf{y} \notin B_{\overline{\rho}}(\mathbf{x}, \varepsilon)$.

For the second containment, consider a basis element $B_{\overline{\rho}}(\mathbf{x}, \varepsilon)$ for the uniform topology on $\reals^J$. Let $U = \prod(x_\alpha - \frac12 \varepsilon, x_\alpha + \frac12 \varepsilon)$ be an open set in the box topology containing $\mathbf{x}$. If $\mathbf{y} \in U$, then $\overline{d}(x_\alpha, y_\alpha) < \frac12 \varepsilon$ for every $\alpha$, and thus $\overline{\rho}(\mathbf{x}, \mathbf{y}) \leq \frac12 \varepsilon < \varepsilon$. Hence $U \subseteq B_{\overline{\rho}}(\mathbf{x}, \varepsilon)$, so the uniform topology is coarser than the box topology.

If $J$ is infinite, then there exists a surjection $\{n_\alpha\}_{\alpha \in J}: J \rightarrow \nats.$ Consider $\mathbf{x} \in \reals^J$ and the basis element $\prod U_\alpha$ for the box topology, where $U_\alpha = (x_\alpha - \frac{1}{n_\alpha}, x_\alpha - \frac{1}{n_\alpha})$. Clearly $\mathbf{x} \in \prod U_\alpha$; we wish to show that there exists no $B_{\overline{\rho}}(\mathbf{y}, \varepsilon)$ containing $\mathbf{x}$ contained in $\prod U_\alpha$. Firstly, if $\mathbf{y} \neq \mathbf{x}$ then for $n_\alpha \geq \frac{1}{|y_\alpha - x_\alpha|}$, $y_\alpha \notin U_\alpha = (x_\alpha - \frac{1}{n_\alpha}, x_\alpha - \frac{1}{n_\alpha})$, hence $\mathbf{y} \notin \prod U_\alpha$. Thus it suffices to consider balls centered at $\mathbf{x}$. Given $\varepsilon > 0$, if $n_\beta > \frac{1}{\varepsilon}$ then 
$$(x_\beta - \frac{1}{n_\beta}, x_\beta + \frac{1}{n_\beta}) \subseteq (x_\beta - \varepsilon, x_\beta + \varepsilon)$$
strictly, so there exists $y_\beta$ in the latter but not the former. Defining $\mathbf{y} \in B_{\overline{\rho}}(\mathbf{x}, \varepsilon)$ by taking this value $y_\beta$ and $y_\alpha = x_\alpha$ for all remaining $\alpha \in J$, we find that $\mathbf{y} \notin \prod U_\alpha$, so $B_{\overline{\rho}}(\mathbf{x}, \varepsilon)$ is not contained in $\prod U_\alpha$. Thus the uniform topology is not coarser than the box topology. $\Box$

Although $\overline{\rho}$ induces a topology different from the product and box topologies for infinite $J$, the following theorem defines a metric that induces the product topology on $\reals^\omega$, the case where $J$ is countably infinite.
\begin{theorem}\label{2.63}
    Let $D: \reals^\omega \times \reals^\omega \rightarrow \reals$ be defined by
    $$D(\mathbf{x}, \mathbf{y}) = \sup \left\{ \frac{\overline{d}(x_i, y_i)}{i} \right\}.$$
    Then $D$ is a metric that induces the product topology on $\reals^\omega$.
\end{theorem}
{\it Proof.} The first two properties of a metric are clearly inherited from $\overline{d}$; to show the triangle inequality,
$$\frac{\overline{d}(x_i, z_i)}{i} \leq \frac{\overline{d}(x_i, y_i)}{i} + \frac{\overline{d}(y_i, z_i)}{i} \leq D(\mathbf{x}, \mathbf{y}) + D(\mathbf{y}, \mathbf{z})$$
for each $i$, hence
$$D(\mathbf{x}, \mathbf{z}) = \sup_i \left\{\frac{\overline{d}(x_i, z_i)}{i}\right\} \leq D(\mathbf{x}, \mathbf{y}) + D(\mathbf{y}, \mathbf{z}).$$
To show that $D$ induces a metric coarser than the product topology, let $\mathbf{x} \in \reals^\omega$ and consider a basis element $B_D(\mathbf{x}, \varepsilon)$. Let $N > \frac{1}{\varepsilon}$, and define the basis element
$$V = (x_1 - \varepsilon, x_1 + \varepsilon) \times \cdots \times (x_N - \varepsilon, x_N + \varepsilon) \times \reals \times \reals \times \cdots$$
for the product topology. We remark that for any $\mathbf{y} \in \reals^\omega$ and $i \geq N$,
$$\frac{\overline{d}(x_i, y_i)}{i} \leq \frac{1}{N},$$
so that
$$D(\mathbf{x}, \mathbf{y}) \leq \max \left\{ \frac{\overline{d}(x_1, y_1)}{1}, \cdots, \frac{\overline{d}(x_N, y_N)}{N}, \frac1N \right\}.$$
For $\mathbf{y} \in V$, we see that $\overline{d}(x_i, y_i) < \varepsilon$ for $1 \leq i \leq N$, so
$$D(\mathbf{x}, \mathbf{y}) < \max \left\{ \frac{\varepsilon}{1}, \cdots, \frac{\varepsilon}{N}, \frac1N \right\} < \varepsilon,$$
and thus $V \subseteq B_D(\mathbf{x}, \varepsilon)$. Conversely, consider a basis element $\prod_{i \in \nats} U_i$ for the product topology, where $U_i = \reals$ for all but $i = i_1, \cdots, i_n$. Given $\mathbf{x} \in U$, choose for each of $i_1, \cdots, i_n$ a $\varepsilon_i \leq 1$ such that $(x_i - \varepsilon_i, x_i + \varepsilon_i) \subseteq U_i$. Then let
$$\varepsilon = \min\{\frac{\varepsilon_i}{i}: i = i_1, \cdots, i_n\}.$$
To show that $B_D(\mathbf{x}, \varepsilon) \subseteq$, $\mathbf{y} \in B_D(\mathbf{x}, \varepsilon)$ implies that for $i = i_1, \cdots, i_n$,
$$\frac{\overline{d}(x_i, y_i)}{i} \leq D(\mathbf{x}, \mathbf{y} < \varepsilon \leq \frac{\varepsilon_i}{i},$$
and thus $\overline{d}(x_i, y_i) < \varepsilon_i \leq 1$ means 
$$\overline{d}(x_i, y_i) = |x_i - y_i| < \varepsilon_i,$$
showing that $\mathbf{y} \in \prod_{i \in \nats} U_i$. $\Box$

The next theorem shows that $\varepsilon$-$\delta$ continuity is equivalent to the topological definition of continuity in any metric space.

\begin{theorem}\label{2.64}
    Let $f: X \rightarrow Y$ where $X, Y$ are metric spaces with metrics $d_X, d_Y$. Then $f$ is continuous if and only if given $x \in X$ and $\varepsilon > 0$, there exists $\delta > 0$ such that
    $$d_X(x, y) < \delta \implies d_Y(f(x), f(y)) < \varepsilon.$$
\end{theorem}
{\it Proof.} Suppose $f$ is continuous; given $x$ and $\varepsilon$, consider the neighbourhood $f^{-1}(B(f(x), \varepsilon))$ of $x$ in $X$. By \hyperref[2.54]{2.54}, there exists $\delta$ such that $B(x, \delta) \subseteq f^{-1}(B(f(x), \varepsilon))$. If $y \in B(x, \delta)$, then $f(y) \in B(f(x), \varepsilon)$.

Conversely, suppose the $\varepsilon$-$\delta$ condition holds; consider any ball $B(f(x), \varepsilon)$. There exists $\delta$ such that $f(B(x, \delta)) \subseteq B(f(x), \varepsilon)$. Thus $f$ is continuous by \hyperref[2.35]{2.35}. $\Box$

\begin{lemma}\label{2.65}
    (The sequence lemma). Suppose $X$ is a topological space and $A \subseteq X$. If there is a sequence of points of $A$ converging to $x$, then $x \in \overline{A}$; the converse holds if $X$ is metrizable.
\end{lemma}
{\it Proof.} Suppose $(x_n)$ is a sequence of points in $A$ converging to $x.$ Since every neighbourhood of $x$ contains points in $A$, $x \in \overline{A}$. Conversely, if $X$ is metrizable and $x \in \overline{A}$, for each $n \in \ints^+$, let $x_n$ be a point in $B_d(x, \frac1n) \cap A$. To show that $(x_n)$ converges to $x$, any neighbourhood of $x$ contains a ball $B_d(x, \varepsilon)$, and taking $N > \frac{1}{\varepsilon}$ ensures that $x_n \in B_d(x, \varepsilon) \subseteq U$ for $n > N$. $\Box$

\begin{theorem}\label{2.66}
    Let $f: X \rightarrow Y$. If $f$ is continuous, then for every convergent sequence $x_n \rightarrow x$ in $X$, $f(x_n)$ converges to $f(x)$. The converse holds if $X$ is metrizable.
\end{theorem}
{\it Proof.} Suppose $f$ is continuous; given $x_n$ converging to $x$, let $V$ be a neighbourhood of $f(x)$. Since $f^{-1}(V)$ is a neighbourhood of $x$, there exists $N$ such that $x_n \in f^{-1}(V)$ for $n > N$, and thus $f(x_n) \in V$ for $n > N$. Conversely, suppose $X$ is a metric space and for every $x_n \rightarrow x$, $f(x_n) \rightarrow f(x)$. Let $A \subseteq X$. If $x \in \overline{A}$, then by the sequence lemma, there is a sequence $(x_n)$ of points of $A$ converging to $x$. Since $f(x_n) \rightarrow f(x)$, $f(x) \in \overline{f(A)}$. Thus $f(\overline{A}) \subseteq \overline{f(A)}$, so $f$ is continuous. $\Box$

The statement of \hyperref[2.65]{2.65} and \hyperref[2.66]{2.66} could be refined; the assumption that $X$ is metrizable is in fact stronger than necessary. It suffices that the countable collection of balls $B_d(x, \frac1n)$ is open in $X$. We formalize this as follows:
\begin{definition}\label{2.67}
    $X$ has a {\it countable basis at} $x$ if there is a countable collection $\{U_n\}$ of neighbourhoods of $x$ such that any neighbourhood $U$ of $x$ contains at least one $U_n$. If $X$ has a countable basis at all $x \in X$, then it satisfies the {\it first countability axiom}.
\end{definition}
All metrizable spaces are first countable, but the converse is not true.

Next, we will proceed with constructing various continuous functions.
\begin{lemma}\label{2.68}
    The addition, subtraction, and multiplication operations are continuous functions $\reals \times \reals \rightarrow \reals$; the quotient operation is a continuous function $\reals \times (\reals - \{0\}) \rightarrow \reals$.
\end{lemma}
{\it Proof.} This result follow rather straightforwardly from the $\varepsilon$-$\delta$ definition of continuity. $\Box$

\begin{theorem}\label{2.69}
    If $f, g: X \rightarrow \reals$ are continuous, then $f+g, f-g$, and $f\cdot g$ are continuous. If $g(x) \neq 0$ for all $x$, then $\frac{f}{g}$ is continuous.
\end{theorem}
{\it Proof.} By \hyperref[2.40]{2.40}, $f \times g$ is continuous. Then $f + g = (+) \circ (f \times g), f - g = (-) \circ (f \times g), f \cdot g = (\cdot) \circ (f \times g)$, and $\frac{f}{g} = (\div) \circ (f \times g)$ are continuous. $\Box$

\begin{definition}\label{2.70}
    Let $f_n:X \rightarrow Y$ be a sequence of functions, where $Y$ is a metric space with metric $d$. $(f_n)$ {\it converges uniformly} to $f: X \rightarrow Y$ if for any $\varepsilon > 0$, there exists $N$ such that
    $$d(f_n(x), f(x)) < \varepsilon$$
    for all $n> N$ and $x \in X$.
\end{definition}

This definition leads to a familiar result in analysis:
\begin{theorem}\label{2.71}
    (Uniform limit theorem). Let $f_n: X \rightarrow Y$ be a sequence of continuous functions, where $Y$ is a metric space. If $(f_n)$ converges uniformly to $f$, then $f$ is continuous.
\end{theorem}
{\it Proof.} For $V$ open in $Y$ and $x_0 \in f^{-1}(V)$, we will construct a neighbourhood $U$ of $x$ such that $f(U) \subseteq V$. Let $B(f(x_0), \varepsilon) \subseteq V$. Then let $N$ be sufficiently large that for all $n \geq N$ and $x \in X$,
$$d(f_n(x), f(x)) < \frac{\varepsilon}{3}.$$
Since $f_N$ is continuous, let $U = f_N^{-1}(B(f_N(x_0), \frac{\varepsilon}{3}))$ is a neighbourhood of $x_0$. Now if $x \in U$, then 
\begin{align*}
    d(f(x), f(x_0)) &\leq d(f(x), f_N(x)) + d(f_N(x), f_N(x_0)) + d(f_N(x_0), f(x_0)), \\
    &< \frac{\varepsilon}{3} + \frac{\varepsilon}{3} + \frac{\varepsilon}{3}, \\
    &< \varepsilon.
\end{align*}
Hence $f(x) \in B(f(x_0), \varepsilon) \subseteq V$, so that $f(U) \subseteq V$. $\Box$

Now that we have sufficiently explored metrizable spaces, we will show that some spaces are not metrizable.
\begin{itemize}
    \item $\reals^\omega$ in the box topology is not metrizable. We will show that the converse of the sequence lemma does not hold, hence the box topology is not metrizable by contraposition. Let
    $$A = \{(x_1, x_2, \cdots) : x_i > 0 \text{ for all } i \in \ints^+\}.$$
    Clearly $\mathbf{0} \in \overline{A}$; suppose $(\mathbf{x}_n)$ is a sequence of points of $A$ converging to $\mathbf{0}$. For each $n$, we may write
    $$\mathbf{x}_n = (x_{1n}, x_{2n}, \cdots)$$
    where each $x_{ij}$ is positive, hence there exists a basis element
    $$(-x_{11}, x_{11}) \times (-x_{22}, x_{22}) \times \cdots$$
    containing $\mathbf{0}$ but containing no points in $(\mathbf{x}_n)$, as for each $n,$ the $n$th coordinate of $\mathbf{x}_n$ lies outside $(-x_{nn}, x_{nn})$. Thus $(\mathbf{x}_n)$ cannot converge to $\mathbf{0}$.

    \item An uncountable Cartesian product of $\reals$ in the product topology is not metrizable. For uncountable $J$, let
    $$A = \{(x_\alpha) \in \reals^J: x_\alpha = 1 \text{ for all but finitely many } \alpha \}.$$
    Given any basis element $\prod U_\alpha$ containing $\mathbf{0}$, let $\alpha_1, \cdots, \alpha_n$ be such that $U_\alpha \neq \reals$. If $(x_\alpha)$ is defined by
    $$x_\alpha = 0 \text{ for } \alpha = \alpha_1, \cdots, \alpha_n \text{ and } x_\alpha = 1 \text{ otherwise },$$
    then $(x_\alpha) \in A \cap \prod U_\alpha$, showing that $\mathbf{0} \in \overline{A}$. Now, suppose there exists a sequence $(\mathbf{a}_n)$ of points in $A$ converging to $\mathbf{0}$. Given $n$, let
    $$J_n = \{\alpha \in J: \mathbf{a}_n (\alpha) \neq 1\},$$
    which is finite. Then $\bigcup_{n \in \ints^+} J_n$ is countable, meaning there exists $\beta \in J - \bigcup_{n \in \ints^+} J_n$. Then, for all $n \in \ints^+,$ $\mathbf{a}_n (\beta) = 1$.

    Now consider the open set $U = \pi_\beta^{-1}(-1, 1)$ in $\reals^J$. $U$ is a neighbourhood of $\mathbf{0}$ that contains no point $\mathbf{a}_n$, contradicting the fact that $(\mathbf{a}_n)$ converges to $\mathbf{0}$.
\end{itemize}

\subsection{The Quotient Topology}
\begin{definition}\label{2.72}
    Let $p: X \rightarrow Y$ be a surjective map between topological spaces. $p$ is a {\it quotient map} if $V$ is open in $Y$ if and only if $p^{-1}(V)$ is open in $X$.
\end{definition}
An equivalent statement for closed sets follows from the fact that
$$f^{-1}(Y-V) = X-f^{-1}(V).$$
\begin{definition}\label{2.73}
    Alternatively, let $U$ be {\it saturated} in $X$ if $U$ is the complete inverse image of some subset of $Y$; that is, for every $x \in U$, $p^{-1}(p(x)) \subseteq U$. Then $p$ is a quotient map if and only if $p$ is continuous and $p$ maps saturated open sets of $X$ to open sets of $Y$.
\end{definition}

Recall that $f: X \rightarrow Y$ is open if for $U$ open in $X$, $f(U)$ is open in $Y$, and closed if for $A$ closed in $X$, $f(A)$ is closed in $Y$. We see that if $p$ is a continuous surjection and either or closed, then $p$ is a quotient map.

Consider $X = [0, 1] \cup [2, 3]$ and $Y = [0, 2]$. Define $p: X \rightarrow Y$ by
$$p(x) = \begin{cases}
    x &\text{for }x \in [0, 1], \\
    x-1 &\text{for }x \in [2, 3].
\end{cases}$$
Then $p$ is clearly surjective, continuous, and closed, so it is a quotient map. However, $p$ is not open since $p([0, 1]) = [0, 1]$ is not open in $Y$. Alternatively, if $A = [0, 1) \cup [2, 3]$, then $\restr{p}{A}: A \rightarrow Y$ is continuous and surjective, but not a quotient map. Indeed, $[2, 3]$ is open in $A$ and saturated with respect to $\restr{p}{A}$, but $\restr{p}{A}([2, 3]) = [1, 2]$ is not open in $Y$.

Now, consider the projection map $\pi_1: \reals \times \reals \rightarrow \reals$, which we know is open, continuous, and surjective, hence a quotient map. However, $\pi_1$ is not closed as $C = \{x\times y: xy = 1\}$ is closed in $\reals$ but $\pi_1(C) = \reals - \{0\}$ is not closed in $\reals$. The restriction of $\pi_1$ to $C \cup \{0 \times 0\}$ is continuous and surjective, but not a quotient map, since $\{0 \times 0\}$ is open in $C \cup \{0 \times 0\}$ and saturated with respect to $q$, but $\pi_1\{0 \times 0\} = \{0\}$ is not open in $\reals$.

Quotient maps can be used to construct topologies, according to the following.
\begin{definition}\label{2.74}
    If $p: X \rightarrow A$ is surjective, then there exists a unique topology $\topo$, called the {\it quotient topology}, on $A$ relative to which $p$ is a quotient map.
\end{definition}
$U$ is open in the quotient topology on $A$ if $p^{-1}(U)$ is open in $X$. As a sanity check, we have $p^{-1}(\varnothing) = \varnothing$ and $p^{-1}(A) = X$. Then
\begin{align*}
    p^{-1} \left( \bigcup_{\alpha \in J}U_\alpha \right) &= \bigcup_{\alpha \in J} p^{-1}(U_\alpha), \\
    p^{-1} \left( \bigcap_{i=1}^n U_i \right) &= \bigcup_{i=1}^n p^{-1}(U_i),
\end{align*}
showing that the quotient topology is closed under arbitrary union and finite intersection.

As an example of a quotient topology, define $p: \reals \rightarrow\{a, b, c\}$ by
$$p(x) = \begin{cases}
    a &\text{if } x > 0, \\
    b &\text{if } x < 0, \\
    c &\text{if } x = 0.
\end{cases}$$
Then the quotient topology induced by $p$ is $\{\varnothing, p^{-1}(0, \infty) = \{a\}, p^{-1}(-\infty, 0) = \{b\}, \{a, b\}, \{a, b, c\}\}$.

An importance instance of the quotient topology is given below.
\begin{definition}\label{2.75}
    Let $X$ be a topological space and $X^*$ a partition of $X$ into disjoint subsets whose union is $X$. Let $p: X \rightarrow X^*$ be a surjection mapping each $x \in X$ to the element of $X^*$ containing $x$. $X^*$ in the quotient topology induced by $p$ is called a quotient space of $X$.
\end{definition}
Given a partition $X^*$, there exists an equivalence relation on $X$ such that the elements of $X^*$ are equivalence classes. Hence $X^*$ may be viewed as an identification or decomposition of equivalent points in $X$. Moreover, since $U \subseteq X^*$ is a collection of equivalence classes, and $p^{-1}(U)$ is the union of equivalence classes, the general open set of $X^*$ is a collection of equivalence classes whose union is open in $X$.

As an example, let $X = \{x \times y: x^2+y^2\leq 1\}$ be the closed unit ball in $\reals^2$, and $X^*$ the partition of $X$ consisting of all one-point sets $\{x \times y\}$ for which $x^2+y^2< 1$ and the set $S^1 = \{x\times y: x^2+y^2 = 1\}$. We construct a homeomorphism between $X^*$ and $S^2 = \{(x, y, z): x^2+y^2+z^2 = 1\}$, the unit 2-sphere in $\reals^3$. Define $f: X^* \rightarrow S^2$ by
$$p(x, y) = (x, y, \sqrt{1 - x^2 - y^2}).$$
Clearly $f$ is continuous and $f^{-1}$ is simply a projection, which is continuous.

As another example, consider $X = [0, 1] \times [0, 1]$ and the partition $X^*$ consisting of all one-point sets $\{x \times y\}$ where $x,y \in (0, 1)$, all two-point sets $\{x \times 0, x\times 1\}$ where $x \in (0, 1)$ or $\{0 \times y, 1 \times y\}$ where $y \in (0, 1)$, and the four-point set $\{0 \times 0, 0 \times 1, 1\times 0, 1 \times 1\}$. $X^*$ is homeomorphic to the torus.

Although quotient maps do not behave perfectly under subspaces, we have the following result.

\begin{theorem}\label{2.76}
    Let $p: X \rightarrow Y$ be a quotient map and $A$ be a subspace of $X$ that is saturated with respect to $p$. If $A$ is either open or closed in $X$, or if $p$ is either an open map or a closed map, then $\restr{p}{A}: A \rightarrow p(A)$ is a quotient map.
\end{theorem}
{\it Proof.} We observe that if $V \subseteq p(A)$, then
$$(\restr{p}{A})^{-1}(V) = \{ x \in A: \restr{p}{A}(x) \in V \} = \{ x \in A: p(x) \in V \} = \{ x \in X: p(x) \in V \} = p^{-1}(V),$$
where the second last equality holds because $A$ is saturated, so $V \subseteq p(A)$ implies $p^{-1}(V) \subseteq A$. Moreover, if $U \subseteq X$, then
$$p(U \cap A) = p(U) \cap p(A).$$
The $\subseteq$ inclusion is clear; the $\supseteq$ inclusion follows from the fact that if $y \in p(U) \cap p(A)$, then $y = p(u) p(a)$ for some $u \in U$ and $a \in A$. Since $A$ is saturated, $u \in p^{-1}(p(a)) \subseteq A$. Then $y = p(u)$ where $u \in U \cap A$.

Suppose $A$ is open. Given a subset $V$ of $p(A)$ such that $(\restr{p}{A})^{-1}(V)$ is open in $A$, $(\restr{p}{A})^{-1}(V)$ is open in $X$ since $A$ is open in $X$. By the remark above, $(\restr{p}{A})^{-1}(V) = p^{-1}(V)$, which is open in $X$. Hence $V$ is open in $Y$ as $p$ is a quotient map. In particular, $V \cap A = V$ is open in $p(A)$.

Alternatively, suppose $p$ is an open map. Given a subset $V$ of $p(A)$ such that $(\restr{p}{A})^{-1}(V) = p^{-1}(V)$ is open in $A$, $p^{-1}(V) = U \cap A$ for some $U$ open in $X$. Then $p(p^{-1}(V)) = V$ as $p$ is surjective, so by the remark above,
$$V = p(p^{-1}(V)) = p(U \cap A) = p(U) \cap p(A).$$
Since $p$ is an open map, $p(U)$ is open in $Y$, and thus $V$ is open in $p(A)$.

The proof when $A$ or $p$ is closed is identical. $\Box$

Composites of quotient maps are again quotient maps, since $U$ is open if and only if $q^{-1}(U)$ is open, if and only if $p^{-1}(q^{-1}(U)) = (q \circ p)^{-1}(U)$ is open.

On the other hand, the Cartesian product of two quotient maps is not necessarily a quotient map. In the special case where $p, q$ are open maps, $p \times q$ is again open, so it is a quotient map.

The Hausdorff condition is also not preserved under quotients; the quotient space $X^*$ of a Hausdorff space may not be Hausdorff. If each element of $X^*$ is closed in $X$, then $X^*$ will be $T_1$, but conditions that imply Hausdorff are more difficult.

The following theorem provides an important criterion for determining when a map from a quotient space is continuous.
\begin{theorem}\label{2.77}
    Let $p:X \rightarrow Y$ be a quotient map. Let $Z$ be a space and $g: X \rightarrow Z$ a map such that for each $y \in Y$, $\restr{g}{p^{-1}(\{y\})}$ is constant. Then $g$ induces a map $f: Y \rightarrow Z$ such that $f \circ p = g$; $f$ is continuous if and only if $g$ is continuous and $f$ is a quotient map if and only if $g$ is a quotient map.
\end{theorem}
{\it Proof.} For each $y \in Y$, $g(p^{-1}(\{y\}))$ is a one-point set in $Z$ since $g$ is constant on this domain. Let $f(y) \in g(p^{-1}(\{y\}))$, uniquely. Thus we have defined $f: Y \rightarrow Z$ such that $f(p(x)) = g(x)$ for all $x \in X$. If $f$ is continuous, then $g = f \circ p$ is a composition of continuous functions. Conversely, if $g$ is continuous, given $V$ open in $Z$, $g^{-1}(V) = p^{-1}(f^{-1}(V))$ is open in $X$. Since $p$ is a quotient map, this implies $f^{-1}(V)$ is open in $Y$, so $f$ is continuous.

If $f$ is a quotient map, then $g$ is the composition of quotient maps. Conversely, if $g$ is a quotient map, it is surjective, so $f$ is surjective. Given $V \subseteq Z$ such that $f^{-1}(V)$ is open in $Y$, $p^{-1}(f^{-1}(V)) = g^{-1}(V)$ is open in $X$ by continuity of $p$. Since $g$ is a quotient map, $V$ is open in $X$. By continuity of $f$, $V$ is open in $X$ if and only if $f^{-1}(V)$ is open in $Y$, so $f$ is a quotient map. $\Box$

\begin{corollary}\label{2.78}
    Let $g: X \rightarrow Z$ be a continuous surjection. Let $X^*$ be defined by
    $$X^* = \{g^{-1}(\{z\}) : z \in Z\},$$
    and given the quotient topology.
    \begin{enumerate}
        \item $g$ induces a continuous bijection $f: X^* \rightarrow Z$, which is a homeomorphism if and only if $g$ is a quotient map.
        \item If $Z$ is Hausdorff, then $X^*$ is Hausdorff.
    \end{enumerate}
\end{corollary}
{\it Proof.} By \hyperref[2.77]{2.77}, $g$ induces a continuous map $f: X^* \rightarrow Z$ and $f$ is clearly bijective by the definition of $X^*$. If $f$ is a homeomorphism, then both $f$ and $p: X \rightarrow X^*$ are quotient maps, so $g = p \circ f$ is a quotient map. Conversely, if $g$ is a quotient map, then $f$ is a quotient map, and since it is bijective, it is a homeomorphism.

If $Z$ is Hausdorff, then given distinct points $x, y$ of $X^*$ their images under $f$ are distinct, so there exist disjoint neighbourhoods $U, V$ of $f(x), f(y)$. Then $f^{-1}(U), f^{-1}(V)$ are disjoint neighbourhoods of $x, y$ in $X^*$. $\Box$

As an example using this corollary, let $X \subseteq \reals^2$ be the union of line segments $[0, 1] \times \{n\}$ for $n \in \ints^+$ and let $Z \subseteq \reals^2$ be the set of all points of the form $x \times \frac{x}{n}$ for $x \in [0, 1]$. Define $g: X \rightarrow Z$ by $g(x \times n) = x \times \frac{x}{n}$; $g$ is clearly surjective and continuous. The quotient space $X^* = \{g^{-1}(\{z\}): z \in Z\}$ is simply the space obtained from $X$ by identifying $\{0 \}\times \ints^+$ to a point. $g$ induces a continuous bijection $f: X^* \rightarrow Z$, which is not homeomorphic. To show this, we will show that $g$ is not a quotient map. Consider $x_n = \frac1n \times n \in X$. $A = \{x_n\}$ is closed, since it has no limit points, and it is saturated with respect to $g$. However, $g(A) = \{\frac1n \times \frac{1}{n^2}: n \in \ints^+\}$ is not closed in $Z$, since $0 \times 0$ is a limit point.

As an example of a product of quotient maps not being a quotient map, let $X = \reals$ and $X^*$ the quotient space obtained by identifying $\ints^+ \subseteq \reals$ to a point $b$; let $p: X \rightarrow X^*$ be the quotient map. Let $i: \rats \rightarrow \rats$ be the identity map, which is a quotient map. We will show that
$$p \times i: X \times \rats \rightarrow X^* \times \rats$$
is not a quotient map. For each $n$, define $c_n = \frac{\sqrt2}{n}$, and consider lines $r_n, s_n$ in $\reals^2$ with respective slopes $1$ and $-1$ passing through $n \times c_n$. Let $U_n$ be the set of points of $X \times \rats$ either lying above both $r_n, s_n$ or lying below both, and between the vertical lines $x = n - \frac14, x = n+ \frac14$. $U_n$ is clearly open in $X \times \rats$ and contains $\{n\}\times \rats$. If $U = \bigcup_{i \in \ints^+} U_i$, then $U$ is open in $X \times \rats$. Since $U$ contains $\ints^+ \times \{q\}$ for each $q \in \rats$, $U$ is saturated with respect to $p \times i$. 

Assume $(p \times i)(U)$ is open in $X^* \times \rats$. Since $\ints^+ \times 0 \subseteq U$, $b \times 0 \in (p \times i)(U)$. Hence $(p \times i)(U)$ contains an open set of the form $W \times I_\delta$, where $W$ is a neighbourhood of $b$ in $X^*$ and $I_\delta = \{y \in \rats: |y| < \delta\}$. Then $p^{-1}(W) \times I_\delta \subseteq U$. Let $n$ be sufficiently large that $c_n < \delta$; since $p^{-1}(W)$ is open in $X$ and contains $\ints^+$, there exists $\varepsilon < \frac14$ such that $(n-\varepsilon, n+\varepsilon) \subseteq p^{-1}(W)$. Then $U$ contains $(n - \varepsilon, n + \varepsilon) \times I_\delta \subseteq X \times \rats$. However, if $y\in \rats$ is such that $|y-c_n| < \frac12 \varepsilon$, then $(n + \frac12 \varepsilon) \times y \in V$ but not in $U$, a contradiction.